%--------------------------------------
%ELECTROTECHNIQUE - SCHEMA DE LIAISON A LA TERRE
%--------------------------------------

%utiliser les environnement \begin{comment} \end{comment} pour mettre en commentaire le préambule une fois la programmation appelée dans le document maître (!ne pas oublier de mettre en commentaire \end{document}!)

\begin{comment}

\documentclass[a4paper, 11pt, twoside, fleqn]{memoir}

\usepackage{AOCDTF}

%--------------------------------------
%CANEVAS
%--------------------------------------

\newcommand\BoxColor{\ifcase\thechapshift blue!30\or brown!30\or pink!30\or cyan!30\or green!30\or teal!30\or purple!30\or red!30\or olive!30\or orange!30\or lime!30\or gray!\or magenta!30\else yellow!30\fi} %définition de la couleur des marqueurs de chapitre

\newcounter{chapshift} %compteur de chapitre du marqueur de chapitre
\addtocounter{chapshift}{-1}
	
\newif\ifFrame %instruction conditionnelle pour les couleurs des pages
\Frametrue

\pagestyle{plain}

% the main command; the mandatory argument sets the color of the vertical box
\newcommand\ChapFrame{%
\AddEverypageHook{%
\ifFrame
\ifthenelse{\isodd{\value{page}}}
  {\backgroundsetup{contents={%
  \begin{tikzpicture}[overlay,remember picture]
  \node[
  	rounded corners=3pt,
    fill=\BoxColor,
    inner sep=0pt,
    rectangle,
    text width=1.5cm,
    text height=5.5cm,
    align=center,
    anchor=north west
  ] 
  at ($ (current page.north west) + (-0cm,-2*\thechapshift cm) $) %nombre négatif = espacement des marqueurs entre les différents chapitres (à régler en fin de rédaction) (4.5cm vaut un espacement équivalement à la hauteur du marqueur, une page peut en contenir 6 avec cet espacement-la mais il est le plus équilibré)
    {\rotatebox{90}{\hspace*{.5cm}%
      \parbox[c][1.2cm][t]{5cm}{%
        \raggedright\textcolor{black}{\sffamily\textbf{\leftmark}}}}};
  \end{tikzpicture}}}
  }
  {\backgroundsetup{contents={%
  \begin{tikzpicture}[overlay,remember picture]
  \node[
  	rounded corners=3pt,
    fill=\BoxColor,
    inner sep=0pt,
    rectangle,
    text width=1.5cm,
    text height=5.5cm,
    align=center,
    anchor=north east
  ] 
  at ($ (current page.north east) + (-0cm,-2*\thechapshift cm) $) %nombre négatif = espacement des marqueurs entre les différents chapitres (à régler en fin de rédaction) (4.5cm vaut un espacement équivalement à la hauteur du marqueur, une page peut en contenir 6 avec cet espacement-la mais il est le plus équilibré)
    {\rotatebox{90}{\hspace*{.5cm}%
      \parbox[c][1.2cm][t]{5cm}{%
        \raggedright\textcolor{black}{\sffamily\textbf{\leftmark}}}}};
  \end{tikzpicture}}}%
  }
  \BgMaterial%
  \fi%
}%
  \stepcounter{chapshift}
}

\renewcommand\chaptermark[1]{\markboth{\thechapter.~#1}{}} %redéfinition du marqueur de chapitre pour ne contenir que le titre du chapitre %à personnaliser selon le nombre de chapitre dans le cours

%--------------------------------------
%corps du document
%--------------------------------------

\begin{document} %corps du document
	\openleft %début de chapitre à gauche

\end{comment}

\begin{wrapfigure}{R}{0pt} %insertion figure dans texte
\begin{tikzpicture}[circuit ee IEC] 
%\DrawGrid{(-7,-5)}{(7,3)} %grille d'aide pour le placement des objets

\fill [gray!50] (-1,-3) -- (5,-3) -- (5,-3.2) -- (-1,-3.2) -- cycle;
\draw [thick] (-1,-3) -- (5,-3);

\node (V) [voltage source] at (0,0) {}; %schéma électrique
\node (C) [make contact] at (2,0) {};
\node (L) [bulb] at (4,0) {};
\draw [brown] (V) to node {} (C);
\draw [brown] (C) to node {} (L);
\draw [blue] (V) |- (2,0.5);
\draw [blue] (2,0.5) -| (L);

\fill [yellow!, decoration=lightning bolt, decorate] (0.4,-0.8) -- (0.9,0); %éclairs
\path [postaction={on each segment={mid arrow=red}}] (0.9,0) -- (1,-0.2) -- (1.4,-0.6)  -- (2,-1) -- (2.3,-2) -- (2.5,-2.9) -- (2.5,-3.1) -- (4,-3.1) -- (4,-3.4);

\draw (2,-1) -- (2.3,-2) -- (2.6,-1) ; %tronc
\draw (2.3,-1) -- (2.3, -0.8); %cou
\draw (2.3,-0.5) circle [radius=0.3cm]; %tête
\draw (0.9,0) -- (1,-0.2) -- (1.4,-0.6)  -- (2,-1) -- (2.6,-1) -- (3.2,-1.4) -- (3.4,-2) -- (3.6,-2); %bras
\draw (2.3,-2.93) -- (2.5,-2.9) -- (2.3, -2) -- (1.6,-2.1) -- (1.2,-2.4) -- (1.1,-2.2); %jambes
\filldraw ([shift=(-10:0.3cm)]2.3,-0.5) arc (-10:150:0.3cm); %casquette
\draw (2.04,-0.34) -- ++ (140:0.3cm); 

\end{tikzpicture}
\end{wrapfigure}

%\end{document}

