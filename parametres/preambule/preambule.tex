%--------------------------------------
%PREAMBULE COMMUN A TOUS LES COURS - A MODIFIER AVEC PRUDENCE
%--------------------------------------

	%--------------------------------------
	%Packages pour le document
	%--------------------------------------

	\usepackage[french]{babel}
	\usepackage{lmodern, marvosym, microtype} %gestion fine des police de caractère
	\usepackage[table, svgnames, dvipsnames, x11names]{xcolor} %gestion des couleurs
	\usepackage[bookmarksnumbered=false, breaklinks, linktoc=all]{hyperref} %référencage
	\usepackage{varioref, memhfixc, url} % amélioration du référencage
	\usepackage{pict2e, picture, multicol, pdflscape, graphicx, eso-pic, preview, graphbox, wrapfig}
	\usepackage{callouts, circledsteps} %annotation d'image
	\usepackage[absolute]{textpos}  % disposition d'images
	\usepackage{authblk, tocbibind, calc} %gestion mise en page
	\usepackage{xifthen, multido, etoolbox, ifpdf} %appel de fonction conditionnelles
	\usepackage{mathtools, amsfonts, amssymb, mathrsfs} %écriture des mathématiques avec référénces
	\usepackage{chemfig, bohr, tikzorbital, chemgreek, expl3, xparse, l3keys2e, xargs, verbatim} %gestion de l'écriture en chimie
	\usepackage{modiagram} %orbitale atomique
	\usepackage{cancel, colortbl, csquotes, mathpazo, soul, caption} %packages nécessaires pour le chargement du package SIunitx
	\usepackage{siunitx} %gestion des unités de physique
	\usepackage[export]{adjustbox}
	\usepackage{animate} %animation d'image
	\usepackage{subcaption, capt-of} %gestion des sous-figures et des légendes communes
	\usepackage{enumitem} %mise en page type code informatique et listes
	\usepackage{verbatim} %mise en page type code informatique et listes
	\usepackage{rotating, mdframed} %rotation
	\usepackage[scale=1,angle=0,opacity=1,contents={}]{background} %gestion de l'arrière-plan 
	\usepackage{xspace, xpatch} %espace après les macros
	\usepackage{pgf, tikz, tikz-qtree, pgfplots, pgfplotstable, schemabloc} %création de figures et schémas
	\usepackage{smartdiagram}
	\usesmartdiagramlibrary{additions}
	\usepackage[siunitx, europeanvoltages, europeancurrents, europeanresistors, americaninductors, europeanports, europeangfsurgearrester, compatibility]{circuitikz}%création de schémas électriques
	\usepackage{pdfpages, scrbase} %inclusion des PDF
	\usepackage{xstring, multirow, booktabs, longtable, makecell, arydshln, cellspace, tabu} %gestion fine des tableaux (sans \ltablex !)
	\usepackage{threeparttable} %notes dans tableaux
	\usepackage[section]{placeins} %force les figures à rester dans leur section
	\usepackage{impnattypo} %règle de typographie française
	\let\newfloat\undefined\usepackage{floatrow} %nouvel environnement flottant

		%--------------------------------------
		%packages aidant à la rédaction
		%--------------------------------------

		\usepackage{lipsum} %insertion LIPSUM
		%\usepackage{showframe} %montre la structure du document
		\usepackage{comment} %commentaire de code sur plusieurs lignes

	%--------------------------------------
	%Packages pour la bibliographie
	%--------------------------------------

	\usepackage[backend=biber, style=numeric, hyperref=auto, citestyle=numeric-comp, autopunct=false]{biblatex}
	\usepackage{biblatex-anonymous}
	\usepackage{csquotes}
	
	%--------------------------------------
	%Pramètrage de la bibliographie
	%--------------------------------------
	
	%\DefineBibliographyStrings{french}{in={dans},inseries={dans}}

	\NewBibliographyString{chapitre}
	\DefineBibliographyStrings{french}{chapitre = {Chap.},}

%--------------------------------------
%paramétrage des packages pour le document
%--------------------------------------

\hypersetup{colorlinks = true, urlcolor = Blue, linkcolor = ForestGreen, citecolor = Red} %paramètrage des couleurs des liens

%\frenchbsetup{StandardLists=true} %liste au format français

\sisetup{%
	locale=FR, %règles de typo française
	detect-all, %on prend la font du document
	group-digits=integer, %le regroupement par 3 chiffres n’a lieu qu’en partie gauche
	free-standing-units, %macro pour les unités existants en dehors des arguments \si et \SI
	group-minimum-digits=5, %groupe si au moins 5 chiffres
	load-configurations=abbreviations % charge les abréviations avec l'argument \SI
	}

	%--------------------------------------
	%paramétrage Tikz-PGF
	%--------------------------------------

	\usetikzlibrary{arrows.meta, pgfplots.dateplot, arrows, shapes.misc, positioning, mindmap, plotmarks, shapes.callouts, fit, matrix, intersections, decorations, decorations.pathmorphing, decorations.pathreplacing, decorations.shapes, decorations.text, decorations.markings, decorations.fractals, decorations.footprints} 
		
	\tikzset{mynode/.style={draw=black, solid, circle, fill=white, inner sep=2pt, thick, text=black}} %pastille d'annotation
	
	\usesmartdiagramlibrary{additions}

	\usepgfplotslibrary{dateplot} %insertion de dates en coordonnées dans les graphiques
	\pgfplotsset{compat=1.16}
	
	\tikzstyle{element}=[rectangle, draw, minimum width=3cm, text centered, text width=2.5cm] %bloc de texte utilisé dans les schémas fléchés

		%--------------------------------------
		%grille d'aide au placement des éléments
		%--------------------------------------
	
		\makeatletter
		\newif\ifmygrid@coordinates
			\tikzset{/mygrid/step line/.style={line width=0.80pt,draw=gray!80},
         		/mygrid/steplet line/.style={line width=0.25pt,draw=gray!80}}
			\pgfkeys{/mygrid/.cd,
         		step/.store in=\mygrid@step,
        			steplet/.store in=\mygrid@steplet,
         		coordinates/.is if=mygrid@coordinates}
			\def\mygrid@def@coordinates(#1,#2)(#3,#4){%
   				\def\mygrid@xlo{#1}%
    				\def\mygrid@xhi{#3}%
    				\def\mygrid@ylo{#2}%
    				\def\mygrid@yhi{#4}%
			}
		\newcommand\DrawGrid[3][]{%
    			\pgfkeys{/mygrid/.cd,coordinates=true,step=1,steplet=0.2,#1}%
    			\draw[/mygrid/steplet line] #2 grid[step=\mygrid@steplet] #3;
    			\draw[/mygrid/step line] #2 grid[step=\mygrid@step] #3;
    			\mygrid@def@coordinates#2#3%
    			\ifmygrid@coordinates%
       			\draw[/mygrid/step line]
        			\foreach \xpos in {\mygrid@xlo,...,\mygrid@xhi} {%
         			(\xpos,\mygrid@ylo) -- ++(0,-3pt)
         			node[anchor=north] {$\xpos$}
        			}
        			\foreach \ypos in {\mygrid@ylo,...,\mygrid@yhi} {%
          			(\mygrid@xlo,\ypos) -- ++(-3pt,0)
					node[anchor=east] {$\ypos$}
        			};
    			\fi%
		}
		\makeatother

%--------------------------------------
%macros
%--------------------------------------

\newcommand*{\superref}[1]{\hyperref[{#1}]{\autoref*{#1} \autopageref*{#1}}} %liens avec page et titre

\definecolor{orangelogo}{RGB}{249,125,0} %couleur orange du logo
\definecolor{bleulogo}{RGB}{11,92,180} %couleur bleue du logo

\newcommand{\circref}[1]{\CircledText{\small\textbf{\ref{#1}}} : } %référence des pastilles

	%--------------------------------------
	%redéfinition des noms
	%--------------------------------------

	\addto\captionsfrench{\renewcommand{\listfigurename}{Liste des figures}} %remplacement des titres automatiques
	\addto\captionsfrench{\renewcommand{\appendixtocname}{Annexes}}
	\addto\captionsfrench{\renewcommand{\appendixpagename}{Annexes}}
 
 	\addto\extrasfrench{%traduction des références automatiques
	\def\sectionautorefname{section}%
    \def\subsectionautorefname{sous-section}%
    \def\figureautorefname{figure}%
    \def\tableautorefname{tableau}%
    \def\exempleautorefname{exemple}%
    \def\exerciceautorefname{exercice}%
    \def\appendixautorefname{annexe}%
	}
  
	\def\frenchfigurename{{\scshape Fig.}} %style des légendes
	\def\frenchtablename{{\scshape Tab.}}
	
	%--------------------------------------
	%paramètres des tableaux
	%--------------------------------------

	\renewcommand\theadfont{\bfseries\small} %style de caractère des en-têtes des tableaux
	\renewcommand\theadalign{cc} %position des en-têtes des tableaux
	\renewcommand\theadgape{} %espacement des en-têtes des tableaux

	\newcommand{\middashrule}{\hdashline\addlinespace} %ligne pointillée de milieu de tableau
	\newcommand{\HRule}{\rule{\linewidth}{0.5mm}} %ligne fine

	%--------------------------------------
	%paramétrage mathématique
	%--------------------------------------

	\newlength{\conditionwd} %environnement de description "muibene" pour les équations et les formules
	\newenvironment{variables}[1][avec\quad] 
  		{#1\tabularx{\textwidth-\widthof{#1}}[t]{
			>{$}l<{$} @{${}:\enspace{}$}l >{ en }l >{(~}l<{~)} l@{}}}	%preambule du tableau à 5 colonnes
		{\endtabularx\\[\belowdisplayskip]}
	
	\newcommand{\pc}[1]{\SI{#1}{\percent}} %nouvelle commande pour afficher les pourcents dans SIunitx

		%--------------------------------------
		%Liste des équations
		%--------------------------------------

		\DeclareNewFloatType{equa}{placement=tbh,fileext=loe} %fileext=extension
		\floatname{equa}{\textsc{\'Eq}} 

		\newcommand{\listofequations}{%
		\listof{equa}{Liste des équations}\addcontentsline{toc}{chapter}{Liste des équations}%
		}

	%--------------------------------------
	%macros pour les listes
	%--------------------------------------

	\newlist{tabdescription}{description}{2} %liste utilisée pour les descriptions dans les tableaux (alignement sur les cellules contenant du texte)
	\setlist[tabdescription, 1]{leftmargin=*, %
	topsep=0ex, %
	parsep=0pt, %
	after=\vspace{-\baselineskip}, %
	before=\vspace{-0.75\baselineskip}}
	
	\setlist[tabdescription, 2]{nosep, %
	leftmargin=*}

	\newlist{tabitemize}{itemize}{1} %liste utilisée dans les tableaux (alignement sur les cellules contenant du texte)
	\setlist[tabitemize]{label=$-$, %
	nosep, %
	leftmargin=*, %
	 topsep=0ex, %
	 parsep=0pt, %
	 after=\vspace{-\baselineskip}, %
	 before=\vspace{-0.75\baselineskip}}

	\newlist{compactitemize}{itemize}{1} %liste compacte et sans marge
	\setlist[compactitemize]{%
	label=$-$, %
	nosep, %
	leftmargin=*}

%--------------------------------------
%Mise en page du document
%--------------------------------------

	%--------------------------------------
	%Marges
	%--------------------------------------

	\setlrmarginsandblock{25mm}{20mm}{*} %réglage marge gauche-droite
	\setulmarginsandblock{20mm}{20mm}{*} %réglage marge haut-bas
	\setheadfoot{\baselineskip}{3\baselineskip} %reglage en hauteur des en-têtes et pied-de-page
	\checkandfixthelayout

	%--------------------------------------
	%En-tête et pied-de-page
	%--------------------------------------

	\newif\ifInvnum\Invnumtrue %énorme prise de tête pour que les numéros de page de l'introduction soient disposés inversement au restant du document

	\pagestyle{plain}{} %réglage de la présence d'en-tête et de pied-de-page 
	\makeevenhead{plain}{}{}{} %en-tête page paire 
	\makeoddhead{plain}{}{}{} %en-tête page impaire 
	\makeevenfoot{plain}{%
		\ifInvnum %instruction conditionelle selon la booleenne Invnum
		\else %si Invnum faux
			{\ifFrame %instruction conditionelle selon la booleenne Frame
				\begin{tikzpicture}
					\draw node [rounded corners=3pt, draw=\BoxColor, fill=\BoxColor, text=black, inner xsep=2ex, inner ysep=5pt]{\sffamily\textbf{\thepage}};
				\end{tikzpicture}
			\else
				\sffamily\textbf{\thepage}
			\fi}
		\fi}
	{\includegraphics[scale=0.03]{logo_compagnons}}
	{%
		\ifInvnum %instruction conditionelle selon la booleenne Invnum
			\sffamily\textbf{\thepage} %si Invnumtrue
		\else
		\fi} %pied-de-page page paire

	\makeoddfoot{plain}{%
		\ifInvnum 
			\sffamily\textbf{\thepage} %si Invnumtrue
		\else
		\fi}
	{\includegraphics[scale=0.03]{logo_compagnons}}
	{%
		\ifInvnum %instruction conditionelle selon la booleenne Invnum
		\else %si Invnum faux
			{\ifFrame %instruction conditionelle selon la booleenne Frame
				\begin{tikzpicture}
					\draw node [rounded corners=3pt, draw=\BoxColor, fill=\BoxColor, text=black, inner xsep=2ex, inner ysep=5pt]{\sffamily\textbf{\thepage}};
				\end{tikzpicture}
			\else
				\sffamily\textbf{\thepage}
			\fi}
		\fi} %pied-de-page page impaire

	%--------------------------------------
	%macros générales
	%--------------------------------------

	\maxsecnumdepth{subsubsection}\setsecnumdepth{subsubsection} %numérotation des sous-sous-sections
	\setcounter{tocdepth}{3} %affichage des sous-sous-sections dans la table des matières
	\setcounter{secnumdepth}{3} %affichage de la numérotation des sous-sous-sections dans la table des matières

	\makeatletter %nouvelle commande conditionnelle interne pour différencier un chapitre en corps de texte et en annexes
		\newcommand{\inappendix}{\fi\expandafter\ifx\@chapapp\appendixname}
	\makeatother

	%--------------------------------------
	%Chapitre
	%--------------------------------------
	
		\makeatletter
	\makechapterstyle{douchy}{%
	
		\setlength{\beforechapskip}{-1cm}
		\setlength{\afterchapskip}{2cm}
		\renewcommand*{\chapnamefont}{\bfseries\sffamily\scriptsize\MakeUppercase}
		\renewcommand*{\chapnumfont}{\mdseries\sffamily}
		\renewcommand*{\chaptitlefont}{\raggedright\Huge\sffamily\bfseries}

		\renewcommand*{\printchaptername}{\raisebox{-\height}{\chapnamefont\@chapapp}}
		\renewcommand*{\chapternamenum}{\hspace{0.3cm}}
		\renewcommand*{\printchapternum}{\raisebox{-\height}{%
			\if\inappendix %instruction conditionelle pour savoir si on se situe dans les annexes ou pas (grosse prise de tête)
				\resizebox{33pt}{!}{\chapnumfont\thechapter} 
			\else
				\resizebox{27pt}{!}{\chapnumfont\thechapter}
			\fi}
		\hspace{0.2cm}}

		\settowidth{\chapindent}{%
		{\printchaptername}\chapternamenum{\printchapternum}\hspace{0.2cm}}

		\renewcommand*{\afterchapternum}{\vspace{-44.5pt}}

		\renewcommand*{\printchaptertitle}[1]{
			\if@mainmatter
				\flushright{\parbox[t]{\linewidth-\chapindent}%
				{\hrule depth 1pt\vspace{3ex}\chaptitlefont ##1}}%
				\vspace{2.75ex}%
				\hrule depth 1pt
			\else
				\parbox[t]{\textwidth}%
				{\hrule depth 1pt\vspace{3ex}\chaptitlefont ##1%
				\vspace{1.25ex}%
				\hrule depth 1pt}
			\fi}

		\renewcommand*{\afterchaptertitle}{\vskip\afterchapskip}
	}
	\makeatother
	\chapterstyle{douchy}
	
	%--------------------------------------
	%Section
	%--------------------------------------
	
	\setsecheadstyle{\huge\sffamily\bfseries\color{black}}

	%--------------------------------------
	%Sous-section
	%--------------------------------------
		
	\renewcommand{\thesubsection}{\arabic{section}. \arabic{subsection}.}
	\setsubsecheadstyle{\Large\sffamily\bfseries\color{black!75}}

	%--------------------------------------
	%Sous-sous-section
	%--------------------------------------

	\setcounter{secnumdepth}{4}
	\renewcommand{\thesubsubsection}{\arabic{section}. \arabic{subsection}. \arabic{subsubsection}.}
	\setsubsubsecheadstyle{\large\sffamily\bfseries\color{gray}}