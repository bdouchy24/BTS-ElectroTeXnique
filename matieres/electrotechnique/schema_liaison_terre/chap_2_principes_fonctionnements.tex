xè%--------------------------------------
%ELECTROTECHNIQUE - SCHEMA DE LIAISON A LA TERRE
%--------------------------------------

%utiliser les environnement \begin{comment} \end{comment} pour mettre en commentaire le préambule une fois la programmation appelée dans le document maître (!ne pas oublier de mettre en commentaire \end{document}!)

\begin{comment}

\documentclass[a4paper, 11pt, twoside, fleqn]{memoir}

\usepackage{AOCDTF}

%--------------------------------------
%CANEVAS
%--------------------------------------

\newcommand\BoxColor{\ifcase\thechapshift blue!30\or brown!30\or pink!30\or cyan!30\or green!30\or teal!30\or purple!30\or red!30\or olive!30\or orange!30\or lime!30\or gray!\or magenta!30\else yellow!30\fi} %définition de la couleur des marqueurs de chapitre

\newcounter{chapshift} %compteur de chapitre du marqueur de chapitre
\addtocounter{chapshift}{-1}
	
\newif\ifFrame %instruction conditionnelle pour les couleurs des pages
\Frametrue

\pagestyle{plain}

% the main command; the mandatory argument sets the color of the vertical box
\newcommand\ChapFrame{%
\AddEverypageHook{%
\ifFrame
\ifthenelse{\isodd{\value{page}}}
  {\backgroundsetup{contents={%
  \begin{tikzpicture}[overlay,remember picture]
  \node[
  	rounded corners=3pt,
    fill=\BoxColor,
    inner sep=0pt,
    rectangle,
    text width=1.3cm,
    text height=5.5cm,
    align=center,
    anchor=north west
  ] 
  at ($ (current page.north west) + (-0cm,-2*\thechapshift cm) $) %nombre négatif = espacement des marqueurs entre les différents chapitres (à régler en fin de rédaction) (4.5cm vaut un espacement équivalement à la hauteur du marqueur, une page peut en contenir 6 avec cet espacement-la mais il est le plus équilibré)
    {\rotatebox{90}{\hspace*{.3cm}%
      \parbox[c][1.2cm][t]{5cm}{%
        \raggedright\textcolor{black}{\sffamily\textbf{\leftmark}}}}};
  \end{tikzpicture}}}
  }
  {\backgroundsetup{contents={%
  \begin{tikzpicture}[overlay,remember picture]
  \node[
  	rounded corners=3pt,
    fill=\BoxColor,
    inner sep=0pt,
    rectangle,
    text width=1.3cm,
    text height=5.5cm,
    align=center,
    anchor=north east
  ] 
  at ($ (current page.north east) + (-0cm,-2*\thechapshift cm) $) %nombre négatif = espacement des marqueurs entre les différents chapitres (à régler en fin de rédaction) (4.5cm vaut un espacement équivalement à la hauteur du marqueur, une page peut en contenir 6 avec cet espacement-la mais il est le plus équilibré)
    {\rotatebox{90}{\hspace*{.3cm}%
      \parbox[c][1.2cm][t]{5cm}{%
        \raggedright\textcolor{black}{\sffamily\textbf{\leftmark}}}}};
  \end{tikzpicture}}}%
  }
  \BgMaterial%
  \fi%
}%
  \stepcounter{chapshift}
}

\renewcommand\chaptermark[1]{\markboth{\thechapter.~#1}{}} %redéfinition du marqueur de chapitre pour ne contenir que le titre du chapitre %à personnaliser selon le nombre de chapitre dans le cours

%--------------------------------------
%corps du document
%--------------------------------------

\begin{document} %corps du document
	\openleft %début de chapitre à gauche

\end{comment}
\chapter{Principes de fonctionnement}
\ChapFrame
 
\section{Terminologie}

La protection contre les contacts indirects dépend principalement des SLT (anciennement régime de neutre) qui sont fonction du branchement du neutre vis-à-vis de la terre et du branchement des masses conductrices vis-à-vis de la terre et du neutre.

\subsection{Définitions usuelles}

\begin{Definition}[Neutre]
 neutre est le point central où sont reliés les trois bobines du secondaire du transformateur HT/BT dans le cas d'un couplage étoile ou zig-zag.
\end{Definition}

\begin{Definition}[Terre]
La terre est la masse conductrice de la terre, dont le potentiel électrique en chaque point est considéré comme égal à zéro.
\end{Definition}

\begin{definition*}[Masse]
Une masse est la partie conductrice d'un appareil électrique susceptible d'être touchée par une personne, qui n'est normalement pas sous tension, mais qui peut le devenir en cas de défaut d'isolement des parties actives de ce matériel (voir \superref{def:masse}).
\end{definition*}

\subsection{Désignations des différents SLT}

\begin{itemize}
\item la première lettre donne la position du neutre de l'installation électrique par rapport à la terre\;,
\item la deuxième lettre donne la position des masses par rapport à la terre où au neutre.
\end{itemize}

\begin{table}
\caption{Désignation des différents schémas de liaisons à la terre}
\begin{tabularx}{\linewidth}{XXX}
\toprule
\thead{Désignation}		& \thead{Branchement du neutre} 	& \thead{Branchement des masses} \\
\midrule
Régime TT						& Neutre relié à la Terre						& Masses reliées à la Terre \\
Régime TN					& Neutre relié à la Terre						& Masses reliées au Neutre \\
Régime IT						& Neutre \og isolé \fg{} (Impédant)	& Masses reliées à la Terre \\
\bottomrule
\end{tabularx}
\end{table}

%\end{document}

