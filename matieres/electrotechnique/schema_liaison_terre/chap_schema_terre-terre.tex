%--------------------------------------
%ELECTROTECHNIQUE - SCHEMA DE LIAISON A LA TERRE
%--------------------------------------

%utiliser les environnement \begin{comment} \end{comment} pour mettre en commentaire le préambule une fois la programmation appelée dans le document maître (!ne pas oublier de mettre en commentaire \end{document}!)

\begin{comment}

\documentclass[a4paper, 11pt, twoside, fleqn]{memoir}

\usepackage{AOCDTF}

%--------------------------------------
%CANEVAS
%--------------------------------------

\newcommand\BoxColor{\ifcase\thechapshift blue!30\or brown!30\or pink!30\or cyan!30\or green!30\or teal!30\or purple!30\or red!30\or olive!30\or orange!30\or lime!30\or gray!\or magenta!30\else yellow!30\fi} %définition de la couleur des marqueurs de chapitre

\newcounter{chapshift} %compteur de chapitre du marqueur de chapitre
\addtocounter{chapshift}{-1}
	
\newif\ifFrame %instruction conditionnelle pour les couleurs des pages
\Frametrue

\pagestyle{plain}

% the main command; the mandatory argument sets the color of the vertical box
\newcommand\ChapFrame{%
\AddEverypageHook{%
\ifFrame
\ifthenelse{\isodd{\value{page}}}
  {\backgroundsetup{contents={%
  \begin{tikzpicture}[overlay,remember picture]
  \node[
  	rounded corners=3pt,
    fill=\BoxColor,
    inner sep=0pt,
    rectangle,
    text width=1.3cm,
    text height=5.5cm,
    align=center,
    anchor=north west
  ] 
  at ($ (current page.north west) + (-0cm,-2*\thechapshift cm) $) %nombre négatif = espacement des marqueurs entre les différents chapitres (à régler en fin de rédaction) (4.5cm vaut un espacement équivalement à la hauteur du marqueur, une page peut en contenir 6 avec cet espacement-la mais il est le plus équilibré)
    {\rotatebox{90}{\hspace*{.3cm}%
      \parbox[c][1.2cm][t]{5cm}{%
        \raggedright\textcolor{black}{\sffamily\textbf{\leftmark}}}}};
  \end{tikzpicture}}}
  }
  {\backgroundsetup{contents={%
  \begin{tikzpicture}[overlay,remember picture]
  \node[
  	rounded corners=3pt,
    fill=\BoxColor,
    inner sep=0pt,
    rectangle,
    text width=1.3cm,
    text height=5.5cm,
    align=center,
    anchor=north east
  ] 
  at ($ (current page.north east) + (-0cm,-2*\thechapshift cm) $) %nombre négatif = espacement des marqueurs entre les différents chapitres (à régler en fin de rédaction) (4.5cm vaut un espacement équivalement à la hauteur du marqueur, une page peut en contenir 6 avec cet espacement-la mais il est le plus équilibré)
    {\rotatebox{90}{\hspace*{.3cm}%
      \parbox[c][1.2cm][t]{5cm}{%
        \raggedright\textcolor{black}{\sffamily\textbf{\leftmark}}}}};
  \end{tikzpicture}}}%
  }
  \BgMaterial%
  \fi%
}%
  \stepcounter{chapshift}
}

\renewcommand\chaptermark[1]{\markboth{\thechapter.~#1}{}} %redéfinition du marqueur de chapitre pour ne contenir que le titre du chapitre %à personnaliser selon le nombre de chapitre dans le cours

%--------------------------------------
%corps du document
%--------------------------------------

\begin{document} %corps du document
	\openleft %début de chapitre à gauche

\end{comment}

\chapter{Schéma Terre-Terre}
\ChapFrame

\section{Caractéristiques générales}

\begin{definition}[Schéma TT]
Schéma de liaison à la terre dans lequel :
\begin{description}
\item[Neutre :] relié à la terre\,;
\item[Masse :] reliées à la terre.
\end{description}
\end{definition}

Dans le SLT TT, le neutre du transformateur HT/BT (point commun) est relié à la terre via la \emph{prise de terre du neutre} \circrefseul{pas:1}. Cette liaison présente une certaine résistance, la \emph{résistance de la prise de terre du neutre} $R_B$ \circrefseul{pas:2}. Sa mise en \oe{}uvre est à charge du fournisseur d'électricité et sa résistance globale doit être inférieure ou égale à \SI{15}{\ohm} \footcite{NF:C13-100-2015}.\\
Les masses sont quant à elles reliées à la terre via la \emph{prise de terre de l'installation électrique} \circrefseul{pas:3}, qui présente aussi une certaine résistance, la \emph{résistance de la prise de terre de l'installation électrique} $R_A$ \circrefseul{pas:4}. Sa mise en \oe{}uvre est à charge du propriétaire de l'installation (voir \superref{subsec:prise_terre_installation_electrique}).\\ 

\section{Schémas de principe}

\begin{figure}[h]
\caption{Installation Terre-Terre}
\begin{subfigure}[t]{0.49\linewidth}
\subcaption{sans défaut d'isolement}
%--------------------------------------
%ELECTROTECHNIQUE - SCHEMA DE LIAISON A LA TERRE
%--------------------------------------

%utiliser les environnement \begin{comment} \end{comment} pour mettre en commentaire le préambule une fois la programmation appelée dans le document maître (!ne pas oublier de mettre en commentaire \end{document}!)

\begin{comment}

\documentclass[a4paper, 11pt, twoside, fleqn]{memoir}

\usepackage{AOCDTF}

%--------------------------------------
%CANEVAS
%--------------------------------------

\newcommand\BoxColor{\ifcase\thechapshift blue!30\or brown!30\or pink!30\or cyan!30\or green!30\or teal!30\or purple!30\or red!30\or olive!30\or orange!30\or lime!30\or gray!\or magenta!30\else yellow!30\fi} %définition de la couleur des marqueurs de chapitre

\newcounter{chapshift} %compteur de chapitre du marqueur de chapitre
\addtocounter{chapshift}{-1}
	
\newif\ifFrame %instruction conditionnelle pour les couleurs des pages
\Frametrue

\pagestyle{plain}

% the main command; the mandatory argument sets the color of the vertical box
\newcommand\ChapFrame{%
\AddEverypageHook{%
\ifFrame
\ifthenelse{\isodd{\value{page}}}
  {\backgroundsetup{contents={%
  \begin{tikzpicture}[overlay,remember picture]
  \node[
  	rounded corners=3pt,
    fill=\BoxColor,
    inner sep=0pt,
    rectangle,
    text width=1.3cm,
    text height=5.5cm,
    align=center,
    anchor=north west
  ] 
  at ($ (current page.north west) + (-0cm,-2*\thechapshift cm) $) %nombre négatif = espacement des marqueurs entre les différents chapitres (à régler en fin de rédaction) (4.5cm vaut un espacement équivalement à la hauteur du marqueur, une page peut en contenir 6 avec cet espacement-la mais il est le plus équilibré)
    {\rotatebox{90}{\hspace*{.3cm}%
      \parbox[c][1.2cm][t]{5cm}{%
        \raggedright\textcolor{black}{\sffamily\textbf{\leftmark}}}}};
  \end{tikzpicture}}}
  }
  {\backgroundsetup{contents={%
  \begin{tikzpicture}[overlay,remember picture]
  \node[
  	rounded corners=3pt,
    fill=\BoxColor,
    inner sep=0pt,
    rectangle,
    text width=1.3cm,
    text height=5.5cm,
    align=center,
    anchor=north east
  ] 
  at ($ (current page.north east) + (-0cm,-2*\thechapshift cm) $) %nombre négatif = espacement des marqueurs entre les différents chapitres (à régler en fin de rédaction) (4.5cm vaut un espacement équivalement à la hauteur du marqueur, une page peut en contenir 6 avec cet espacement-la mais il est le plus équilibré)
    {\rotatebox{90}{\hspace*{.3cm}%
      \parbox[c][1.2cm][t]{5cm}{%
        \raggedright\textcolor{black}{\sffamily\textbf{\leftmark}}}}};
  \end{tikzpicture}}}%
  }
  \BgMaterial%
  \fi%
}%
  \stepcounter{chapshift}
}

\renewcommand\chaptermark[1]{\markboth{\thechapter.~#1}{}} %redéfinition du marqueur de chapitre pour ne contenir que le titre du chapitre %à personnaliser selon le nombre de chapitre dans le cours

%--------------------------------------
%corps du document
%--------------------------------------

\begin{document} %corps du document
	\openleft %début de chapitre à gauche

\end{comment}

\begin{circuitikz}[circuit ee IEC relay]
%\DrawGrid{(-1,-5)}{(9,3)} %grille d'aide pour le placement des objets

%sol

\fill [gray!50] (-1,-3.5) -- (5.5,-3.5) -- (5.5,-3.7) -- (-1,-3.7) -- cycle;
\draw [thick] (-1,-3.5) -- (5.5,-3.5);

%alimentation

\node (T1) [oosourcetransshape,prim=delta,sec=wye] at (0,0) {};
\node (D1) [make contact=point left, circuit breaker={point left}, tiny circuit symbols] at (1,0.45) {};
\node (D2) [make contact=point left, circuit breaker={point left}, tiny circuit symbols] at (1,0.15) {};
\node (D3) [make contact=point left, circuit breaker={point left}, tiny circuit symbols] at (1,-0.15) {};
\node (D4) [make contact=point left, circuit breaker={point left}, tiny circuit symbols] at (1,-0.45) {};

\draw [rounded corners=0.2cm] (1.4, 0.6) rectangle (1.8,-0.6);
\draw [dashed, rounded corners, thin]  (D1) to (1,-0.8) to (1.6,-0.8) to (1.6,-0.6);

\draw [brown] (-1,0.3) to (-0.5,0.3) to node {} (T1.prim1);
\draw [black] (-1,0) to (-0.5,0) to node {} (T1.prim2);
\draw [gray] (-1,-0.3) to (-0.5,-0.3) to node {} (T1.prim3);

\draw [brown] (5.5,0.45) to (D1) to (0.5,0.45) to (T1.sec1);
\draw [black] (5.5,0.15) to (D2) to (0.5,0.15) to (T1.sec2);
\draw [gray] (5.5,-0.15) to (D3) to (0.5,-0.15) to (T1.sec3);
\draw [blue] (5.5,-0.45) to (D4) to (0.5,-0.45) to (T1.sec4);

%neutre/terre

\node (RN) [resistor, label=$R_B$, rotate=90, tiny circuit symbols] at (0,-2.7) {};
\node (G1) [tlground] at (0,-3.9) {};
\draw [green!] (G1) to node {} (RN) ; 
\draw [green!] (RN) to (0,-0.5) to node {} (T1.sec4) ; 
\draw [dashed, yellow!] (G1) to node {} (RN) ;
\draw [dashed, yellow!] (RN) to (0,-0.5) to node {} (T1.sec4) ;

\node (G1) [tlground] at (0,-3.9) {};
\node (T1) [oosourcetransshape, prim=delta,sec=wye] at (0,0) {};

\node (RT) [resistor, label=$R_A$, rotate=90, tiny circuit symbols] at (2.5,-2.7) {};
\node (G2) [tlground] at (2.5,-3.9) {};
\draw [green!] (RT) to (G2); 
\draw [dashed, yellow!] (RT) to (G2);
\node (G2) [tlground] at (2.5,-3.9) {};

%appareil 1

\node (C1) [circ, scale=0.5] at (2.5,-1.8) {};
\node (C2) [circ, scale=0.5] at (2.4,0.45) {};
\node (C3) [circ, scale=0.5] at (2,-0.45) {};
\node (L) [bulb, info=1, rotate=270] at (2.4,-1.5) {};

\draw [green!] (C1) to node {} (RT); 
\draw [dashed, yellow!] (C1) to node {} (RT); 

\draw (2.1,-1.2) rectangle (2.7,-1.8);
\draw [brown] (C2) to node {} (L);
\draw [blue] (C3) to (2,-2) to (2.4,-2) to node {} (L);

%appareil 2

\node (C5) [circ, scale=0.5] at (3.7,-1.8) {};
\node (C6) [circ, scale=0.5] at (3.6,0.45) {};
\node (C7) [circ, scale=0.5] at (3.2,-0.45) {};
\node (C8) [circ, scale=0.5] at (2.5,-2.1) {};
\node (L) [bulb, info=2, rotate=270] at (3.6,-1.5) {};

\draw [green!] (C8) -| (C5); 
\draw [dashed, yellow!] (C8) -| (C5); 

\draw (3.3,-1.2) rectangle (3.9,-1.8);
\draw [brown] (C6) to node {} (L);
\draw [blue] (C7) to (3.2,-2) to (3.6,-2) to node {} (L);

%appareil 3

\node (C9) [circ, scale=0.5] at (4.9,-1.8) {};
\node (C10) [circ, scale=0.5] at (4.8,0.45) {};
\node (C11) [circ, scale=0.5] at (4.4,-0.45) {};
\node (C12) [circ, scale=0.5] at (2.5,-2.2) {};
\node (L) [bulb, info=3, rotate=270] at (4.8,-1.5) {};

\draw [green!] (C12) -| (C9); 
\draw [dashed, yellow!] (C12) -| (C9); 

\draw (4.5,-1.2) rectangle (5.1,-1.8);
\draw [brown] (C10) to node {} (L);
\draw [blue] (C11) to (4.4,-2) to (4.8,-2) to node {} (L);
%chemin courant

\callout{1,-2}{\cstep\label{pas:1}}{0.2,-3.8};
\callout{-0.5,-1}{\cstep\label{pas:2}}{-0.1,-2.4};
\callout{1.5,-3}{\cstep\label{pas:3}}{2.2,-3.8};
\callout{4,-2.6}{\cstep\label{pas:4}}{2.6,-2.6};
\startcstep %remet les compteurs des légendes en pastille à zéro
\end{circuitikz}
%\end{document}


\end{subfigure}
\begin{subfigure}[t]{0.49\linewidth}
\subcaption{avec défaut d'isolement}
%--------------------------------------
%ELECTROTECHNIQUE - SCHEMA DE LIAISON A LA TERRE
%--------------------------------------

%utiliser les environnement \begin{comment} \end{comment} pour mettre en commentaire le préambule une fois la programmation appelée dans le document maître (!ne pas oublier de mettre en commentaire \end{document}!)

\begin{comment}

\documentclass[a4paper, 11pt, twoside, fleqn]{memoir}

\usepackage{AOCDTF}

%--------------------------------------
%CANEVAS
%--------------------------------------

\newcommand\BoxColor{\ifcase\thechapshift blue!30\or brown!30\or pink!30\or cyan!30\or green!30\or teal!30\or purple!30\or red!30\or olive!30\or orange!30\or lime!30\or gray!\or magenta!30\else yellow!30\fi} %définition de la couleur des marqueurs de chapitre

\newcounter{chapshift} %compteur de chapitre du marqueur de chapitre
\addtocounter{chapshift}{-1}
	
\newif\ifFrame %instruction conditionnelle pour les couleurs des pages
\Frametrue

\pagestyle{plain}

% the main command; the mandatory argument sets the color of the vertical box
\newcommand\ChapFrame{%
\AddEverypageHook{%
\ifFrame
\ifthenelse{\isodd{\value{page}}}
  {\backgroundsetup{contents={%
  \begin{tikzpicture}[overlay,remember picture]
  \node[
  	rounded corners=3pt,
    fill=\BoxColor,
    inner sep=0pt,
    rectangle,
    text width=1.3cm,
    text height=5.5cm,
    align=center,
    anchor=north west
  ] 
  at ($ (current page.north west) + (-0cm,-2*\thechapshift cm) $) %nombre négatif = espacement des marqueurs entre les différents chapitres (à régler en fin de rédaction) (4.5cm vaut un espacement équivalement à la hauteur du marqueur, une page peut en contenir 6 avec cet espacement-la mais il est le plus équilibré)
    {\rotatebox{90}{\hspace*{.3cm}%
      \parbox[c][1.2cm][t]{5cm}{%
        \raggedright\textcolor{black}{\sffamily\textbf{\leftmark}}}}};
  \end{tikzpicture}}}
  }
  {\backgroundsetup{contents={%
  \begin{tikzpicture}[overlay,remember picture]
  \node[
  	rounded corners=3pt,
    fill=\BoxColor,
    inner sep=0pt,
    rectangle,
    text width=1.3cm,
    text height=5.5cm,
    align=center,
    anchor=north east
  ] 
  at ($ (current page.north east) + (-0cm,-2*\thechapshift cm) $) %nombre négatif = espacement des marqueurs entre les différents chapitres (à régler en fin de rédaction) (4.5cm vaut un espacement équivalement à la hauteur du marqueur, une page peut en contenir 6 avec cet espacement-la mais il est le plus équilibré)
    {\rotatebox{90}{\hspace*{.3cm}%
      \parbox[c][1.2cm][t]{5cm}{%
        \raggedright\textcolor{black}{\sffamily\textbf{\leftmark}}}}};
  \end{tikzpicture}}}%
  }
  \BgMaterial%
  \fi%
}%
  \stepcounter{chapshift}
}

\renewcommand\chaptermark[1]{\markboth{\thechapter.~#1}{}} %redéfinition du marqueur de chapitre pour ne contenir que le titre du chapitre %à personnaliser selon le nombre de chapitre dans le cours

%--------------------------------------
%corps du document
%--------------------------------------

\begin{document} %corps du document
	\openleft %début de chapitre à gauche

\end{comment}
\begin{circuitikz}[circuit ee IEC relay]
%\DrawGrid{(-1,-5)}{(9,3)} %grille d'aide pour le placement des objets

%sol

\fill [gray!50] (-1,-3.5) -- (5.5,-3.5) -- (5.5,-3.7) -- (-1,-3.7) -- cycle;
\draw [thick] (-1,-3.5) -- (5.5,-3.5);

%alimentation

\node (T1) [oosourcetransshape,prim=delta,sec=wye] at (0,0) {};
\node (D1) [make contact=point left, circuit breaker={point left}, activated, tiny circuit symbols] at (1,0.45) {};
\node (D2) [make contact=point left, circuit breaker={point left}, activated, tiny circuit symbols] at (1,0.15) {};
\node (D3) [make contact=point left, circuit breaker={point left}, activated, tiny circuit symbols] at (1,-0.15) {};
\node (D4) [make contact=point left, circuit breaker={point left}, activated, tiny circuit symbols] at (1,-0.45) {};

\draw [rounded corners=0.2cm] (1.4, 0.6) rectangle (1.8,-0.6);
\draw [dashed, rounded corners, ultra thin]  (1, 0.45) to (1,-0.8) to (1.6,-0.8) to (1.6,-0.6);

\draw [brown, ultra thin] (-1,0.3) to (-0.5,0.3) to node {} (T1.prim1);
\draw [black, ultra thin] (-1,0) to (-0.5,0) to node {} (T1.prim2);
\draw [gray, ultra thin] (-1,-0.3) to (-0.5,-0.3) to node {} (T1.prim3);

\draw [brown, ultra thin] (5.5,0.45) to (D1) to (0.5,0.45) to node {} (T1.sec1);
\draw [black, ultra thin] (5.5,0.15) to (D2) to (0.5,0.15) to node {} (T1.sec2);
\draw [gray, ultra thin] (5.5,-0.15) to (D3) to (0.5,-0.15) to node {} (T1.sec3);
\draw [blue, ultra thin] (5.5,-0.45) to (D4) to (0.5,-0.45) to node {} (T1.sec4);

%neutre/terre

\node (RN) [resistor, label=$R_B$, rotate=90, tiny circuit symbols] at (0,-2.7) {};
\node (G1) [tlground] at (0,-3.9) {};
\draw [green!, thick] (G1) to node {} (RN) ; 
\draw [green!, thick] (RN) to (0,-0.5) to node {} (T1.sec4) ; 
\draw [dashed, yellow!, thick] (G1) to node {} (RN) ;
\draw [dashed, yellow!, thick] (RN) to (0,-0.5) to node {} (T1.sec4) ;

\node (G1) [tlground] at (0,-3.9) {};
\node (T1) [oosourcetransshape, prim=delta,sec=wye] at (0,0) {};

\node (RT) [resistor, label=$R_A$, rotate=90, tiny circuit symbols] at (2.5,-2.7) {};
\node (G2) [tlground] at (2.5,-3.9) {};
\draw [green!, thick] (RT) to (G2); 
\draw [dashed, yellow!, thick] (RT) to (G2);
\node (G2) [tlground] at (2.5,-3.9) {};

%appareil 1

\node (C1) [circ, scale=0.5] at (2.5,-1.8) {};
\node (C2) [circ, scale=0.5] at (2.4,0.45) {};
\node (C3) [circ, scale=0.5] at (2,-0.45) {};
\node (L) [bulb, info=1, rotate=270] at (2.4,-1.5) {};

\draw [green!, thick] (C1) to node {} (RT); 
\draw [dashed, yellow!, thick] (C1) to node {} (RT); 

\draw [thick] (2.1,-1.2) rectangle (2.7,-1.8);
\draw [brown, thick] (T1.sec1) to (0.5,0.45) to (D1) to (C2) to node {} (L);
\draw [blue, ultra thin] (C3) to (2,-2) to (2.4,-2) to node {} (L);

%appareil 2

\node (C5) [circ, scale=0.5] at (3.7,-1.8) {};
\node (C6) [circ, scale=0.5] at (3.6,0.45) {};
\node (C7) [circ, scale=0.5] at (3.2,-0.45) {};
\node (C8) [circ, scale=0.5] at (2.5,-2.1) {};
\node (L) [bulb, info=2, rotate=270] at (3.6,-1.5) {};

\draw [green!, ultra thin] (C8) -| (C5); 
\draw [dashed, yellow!, ultra thin] (C8) -| (C5); 

\draw (3.3,-1.2) rectangle (3.9,-1.8);
\draw [brown, ultra thin] (C6) to node {} (L);
\draw [blue, ultra thin] (C7) to (3.2,-2) to (3.6,-2) to node {} (L);

%appareil 3

\node (C9) [circ, scale=0.5] at (4.9,-1.8) {};
\node (C10) [circ, scale=0.5] at (4.8,0.45) {};
\node (C11) [circ, scale=0.5] at (4.4,-0.45) {};
\node (C12) [circ, scale=0.5] at (2.5,-2.2) {};
\node (L) [bulb, info=3, rotate=270] at (4.8,-1.5) {};

\draw [green!, ultra thin] (C12) -| (C9); 
\draw [dashed, yellow!, ultra thin] (C12) -| (C9); 

\draw (4.5,-1.2) rectangle (5.1,-1.8);
\draw [brown, ultra thin] (C10) to node {} (L);
\draw [blue, ultra thin] (C11) to (4.4,-2) to (4.8,-2) to node {} (L);
%chemin courant

%\path [postaction={on each segment={mid arrow=red}}]  (3.8,-1.8) -- (RT) -- (G2) -- (3.8,-4.4) -- (1.9,-4.4) -- (0,-4.4) -- (G1) -- (RN) -- (0,-0.5) -- (T1.sec4); 

\fill [yellow!, decoration=lightning bolt, decorate] (2.4,-1.2) -- ++ (0.5,0.8); %éclairs
\path [postaction={on each segment={mid arrow=red}}] (2.4,-1.2) -- (2.7,-1.2) -- (2.7,-1.8) -- (C1) -- (RT) -- (G2) -- (2.5,-4.2) -- (1.666,-4.2) -- (0.88888,-4.2)  -- (0,-4.2) -- (G1) -- (RN) -- (0,-0.5) -- (T1.sec4); 
\end{circuitikz}

%\end{document}


\end{subfigure}
\end{figure}

En cas de défaut d'isolement sur les masses métalliques, le courant de défaut $I_d$ dispose d'un chemin, via la terre, pour revenir au poste de transformateur HT/BT. Cela forme la \emph{boucle de défaut}.\\
Dans les calculs, il faut tenir compte de la \emph{résistance de défaut} $R_d$ \circrefseul{pas:1} qui prend en compte la nature du défaut d'isolement (franc ou non-franc) et la résistance de la carcasse métallique.\\

\pagebreak

%--------------------------------------
%ELECTROTECHNIQUE - SCHEMA DE LIAISON A LA TERRE
%--------------------------------------

%utiliser les environnement \begin{comment} \end{comment} pour mettre en commentaire le préambule une fois la programmation appelée dans le document maître (!ne pas oublier de mettre en commentaire \end{document}!)

\begin{comment}

\documentclass[a4paper, 11pt, twoside, fleqn]{memoir}

\usepackage{AOCDTF}

%--------------------------------------
%CANEVAS
%--------------------------------------

\newcommand\BoxColor{\ifcase\thechapshift blue!30\or brown!30\or pink!30\or cyan!30\or green!30\or teal!30\or purple!30\or red!30\or olive!30\or orange!30\or lime!30\or gray!\or magenta!30\else yellow!30\fi} %définition de la couleur des marqueurs de chapitre

\newcounter{chapshift} %compteur de chapitre du marqueur de chapitre
\addtocounter{chapshift}{-1}
	
\newif\ifFrame %instruction conditionnelle pour les couleurs des pages
\Frametrue

\pagestyle{plain}

% the main command; the mandatory argument sets the color of the vertical box
\newcommand\ChapFrame{%
\AddEverypageHook{%
\ifFrame
\ifthenelse{\isodd{\value{page}}}
  {\backgroundsetup{contents={%
  \begin{tikzpicture}[overlay,remember picture]
  \node[
  	rounded corners=3pt,
    fill=\BoxColor,
    inner sep=0pt,
    rectangle,
    text width=1.3cm,
    text height=5.5cm,
    align=center,
    anchor=north west
  ] 
  at ($ (current page.north west) + (-0cm,-2*\thechapshift cm) $) %nombre négatif = espacement des marqueurs entre les différents chapitres (à régler en fin de rédaction) (4.5cm vaut un espacement équivalement à la hauteur du marqueur, une page peut en contenir 6 avec cet espacement-la mais il est le plus équilibré)
    {\rotatebox{90}{\hspace*{.3cm}%
      \parbox[c][1.2cm][t]{5cm}{%
        \raggedright\textcolor{black}{\sffamily\textbf{\leftmark}}}}};
  \end{tikzpicture}}}
  }
  {\backgroundsetup{contents={%
  \begin{tikzpicture}[overlay,remember picture]
  \node[
  	rounded corners=3pt,
    fill=\BoxColor,
    inner sep=0pt,
    rectangle,
    text width=1.3cm,
    text height=5.5cm,
    align=center,
    anchor=north east
  ] 
  at ($ (current page.north east) + (-0cm,-2*\thechapshift cm) $) %nombre négatif = espacement des marqueurs entre les différents chapitres (à régler en fin de rédaction) (4.5cm vaut un espacement équivalement à la hauteur du marqueur, une page peut en contenir 6 avec cet espacement-la mais il est le plus équilibré)
    {\rotatebox{90}{\hspace*{.3cm}%
      \parbox[c][1.2cm][t]{5cm}{%
        \raggedright\textcolor{black}{\sffamily\textbf{\leftmark}}}}};
  \end{tikzpicture}}}%
  }
  \BgMaterial%
  \fi%
}%
  \stepcounter{chapshift}
}

\renewcommand\chaptermark[1]{\markboth{\thechapter.~#1}{}} %redéfinition du marqueur de chapitre pour ne contenir que le titre du chapitre %à personnaliser selon le nombre de chapitre dans le cours

%--------------------------------------
%corps du document
%--------------------------------------

\begin{document} %corps du document
	\openleft %début de chapitre à gauche

\end{comment}

\begin{figure}[h]
\caption{Boucle de défaut du courant $I_d$ sur L1}
\begin{circuitikz}[circuit ee IEC relay]
%\DrawGrid{(-1,-5)}{(9,3)} %grille d'aide pour le placement des objets

%alimentation

\node (D1) [make contact=point left, circuit breaker={point left}, tiny circuit symbols, activated] at (1,0.45) {};
\node (T1) [oosourcetransshape, prim=delta,sec=wye] at (0,0) {};


%neutre/terre

\node (RN) [R, label=$R_B$, rotate=90, tiny circuit symbols] at (0,-2.7) {};
\node (G1) [tlground] at (0,-3.9) {};
\draw [green!, thick] (G1) to node {} (RN) ; 
\draw [green!, thick] (RN) to (0,-0.5) to node {} (T1.sec4) ; 
\draw [dashed, yellow!, thick] (G1) to node {} (RN) ;
\draw [dashed, yellow!, thick] (RN) to (0,-0.5) to node {} (T1.sec4) ;

\node (RT) [resistor, rotate=90, tiny circuit symbols, label=$R_A$] at (2.5,-2.7) {};
\draw[-triangle 45, red] (2.8,-2) -- (2.8,-1) node[right,midway] {$U_d$};
\node (G2) [tlground] at (2.5,-3.9) {};
\draw [green!, thick] (RT) to (G2); 
\draw [dashed, yellow!, thick] (RT) to (G2);
\node (G2) [tlground] at (2.5,-3.9) {};
\draw [green!, thick] (G1) to (0,-4.2) to (2.5,-4.2) to (G2);
\draw [dashed, yellow!, thick] (G1) -- (0,-4.2) -- (2.5,-4.2) node [midway,below] {\color{black}$I_d$} -- (G2);
\node (G1) [tlground] at (0,-3.9) {};
\node (G2) [tlground] at (2.5,-3.9) {};

%appareil 1

\node (C2) [circ, scale=0.5] at (2.5,0.45) {};
\node (RD) [resistor, label=$R_d$, rotate=90, tiny circuit symbols] at (2.5,-1.5) {};

\draw [green!, thick] (RD) to (RT); 
\draw [dashed, yellow!, thick] (RD) to (RT); 

\draw [brown, thick] (T1.sec1) to (0.5,0.45) to (D1) to (C2) to (RD);
\node (T1) [oosourcetransshape, prim=delta,sec=wye] at (0,0) {};

%chemin courant

\fill [yellow!, decoration=lightning bolt, decorate] (2.5,-1.2) -- ++ (0.5,0.8); %éclairs
\path [postaction={on each segment={mid arrow=red}}]  (T1.sec1) -- (0.5,0.45) -- (D1) -- (C2) -- (RD) -- (RT) -- (G2) -- (2.5,-4.2) -- (1.666,-4.2) -- (0.88888,-4.2)  -- (0,-4.2) -- (G1) -- (RN) -- (0,-0.5) -- (T1.sec4); 

\callout{1,-0.5}{\cstep\label{pas:1}}{2.4,-1.2};


\end{circuitikz}
\end{figure}



%\end{document}



L'intensité de courant $I_d$ vaut alors :
\begin{formule}[Courant de défaut $I_d$]
\begin{align}
		I_d &= \frac{V}{R_{B}+R_{A}+R_{d}}
\end{align}
\end{formule}

\begin{textvariables}
V								& tension							& volt			& \volt					& 	Différence de potentiel entre les masses métalliques et la terre 	\\
R_{B}						& résistance						& ohm			& \ohm					& 	Résistance de la prise de terre du neutre 	\\
R_{A}						& résistance						& ohm			& \ohm					& 	Résistance de la prise de terre de l'installation électrique 	\\
R_{d}						& résistance						& ohm			& \ohm					& 	Résistance de défaut 	d'isolement \\
\end{textvariables}

Le courant de défaut $I_d$ fera alors apparaître une \emph{tension de défaut} $U_d$ entre la masse métallique et la terre. Pour satisfaire aux normes de sécurité de la NF C15-100, il est imposé (voir \superref{subsec:prise_terre_installation_electrique}) :

\begin{formule}[Tension de défaut $U_d$]
\begin{align}
		U_d &= R_{A} \cdot I_{d} \\
			   &< U_L
\end{align}
\end{formule}

\begin{textvariables}
R_{A}						& résistance						& ohm			& \ohm					& 	Résistance de la prise de terre de l'installation électrique 	\\
I_{d}							& intensité							& ampère		& \ampere				& 	Courant de défaut d'isolement \\
U_{L}						& tension							& volt			& \volt					& 	Tension de sécurité du local avec :
\begin{description}[nosep, leftmargin=*]
\item[Local sec :] $U_{L}=\SI{50}{\volt}$
\item[Local humide :] $U_{L}=\SI{25}{\volt}$
\end{description} \\
\end{textvariables}

Il est donc nécessaire de limiter $U_d$ à la valeur suivante (voir \superref{form:resistance_prise_terre}) :

\begin{formule}[Calibre du DDR $I_{\Delta n}$]
\begin{align}
		I_{\Delta n} &< \frac{U_{L}}{R_{A}}
\end{align}
\end{formule}

\begin{textvariables}
U_{L}						& tension							& volt			& \volt					& 	Tension de sécurité du local avec :
\begin{description}[nosep, leftmargin=*]
\item[Local sec :] $U_{L}=\SI{50}{\volt}$
\item[Local humide :] $U_{L}=\SI{25}{\volt}$
\end{description} \\
R_{A}						& résistance						& ohm			& \ohm					& 	Résistance de la prise de terre de l'installation électrique 	\\
\end{textvariables}

\begin{exemple}[Calcul du calibre du DDR $I_{\Delta n}$]
Si on considère que le transformateur est un transformateur $\SI{20}{\kilo\volt}/\SI{400}{\volt}$, que $R_A=\SI{20}{\ohm}$, que $R_B=\SI{10}{\ohm}$ et que $R_d$ est négligée, on peut déduire que le courant de défaut $I_d$ vaut :
\begin{align*}
		I_d 	&= \frac{V}{R_{B}+R_{A}} \\
				&=\frac{400}{20+10} \\
				&= \SI{13,33}{\ampere} \\
\end{align*}
Si une personne touche une masse des récepteurs en défaut, elle sera soumise à une tension de défaut $U_d$ :
\begin{align*}
		U_d 	&= R_{A} \cdot I_{d} \\
				&=20 \cdot 13,33 \\
				&= \SI{266,6}{\volt}
\end{align*}
La tension de défaut $U_d$ est dangereuse quelle que soit la tension limite choisie :
\begin{itemize}
\item coupure la plus rapide possible\,;
\item protection des personnes.
\end{itemize}
~\\
\begin{minipage}[t]{0.5\linewidth}
Dans le cas d'un local sec :
\begin{align*}
	I_{\Delta n} 	&< \frac{U_{L}}{R_{A}} \\
						&< \frac{50}{20} \\
						&< \SI{2,5}{\ampere}
\end{align*}
\end{minipage}
\hfill
\begin{minipage}[t]{0.5\linewidth}
Dans le cas d'un local humide :
\begin{align*}
	I_{\Delta n} 	&< \frac{U_{L}}{R_{A}} \\
						&< \frac{25}{20} \\
						&< \SI{1,25}{\ampere}
\end{align*}
\end{minipage}
~\\
D'après le tableau situé en \superref{tab:temps_coupure_DDR}, le DDR doit présenter un temps de coupure de moins de \SI{70}{\milli\second} avec une tension de défaut $U_d$ de \SI{266,6}{\volt} :

\begin{table}[h]
\begin{tabularx}{\linewidth}{X cccccccc}
\toprule
Tension nominale		& \multicolumn{2}{c}{$\SI{50}{\volt}<U_0\leq\SI{120}{\volt}$} 	& \multicolumn{2}{c}{$\SI{120}{\volt}<U_0\leq\SI{230}{\volt}$} & \multicolumn{2}{c}{$\SI{230}{\volt}<U_0\leq\SI{400}{\volt}$}		& \multicolumn{2}{c}{$U_0>\SI{400}{\volt}$}\\
\midrule
Type de courant		& alternatif	& continu	& alternatif	& continu	& alternatif	& continu	& alternatif	& continu \\
\addlinespace
Schéma TN/IT	& \SI{0,8}{\second}	&	\SI{5}{\second}	&	\SI{0,4}{\second}	&	\SI{5}{\second}	&	\SI{0,2}{\second}	&	\SI{0,4}{\second}	&	\SI{0,1}{\second}	&	\SI{0,1}{\second} \\	
\addlinespace
Schéma TT	& \SI{0,3}{\second}	&	\SI{5}{\second}	&	\SI{0,2}{\second}	&	\SI{0,4}{\second}	&	\cellcolor{green}\SI{0,07}{\second}	&	\SI{0,2}{\second}	&	\SI{0,04}{\second}	&	\SI{0,1}{\second} \\	
\bottomrule
\end{tabularx}
\end{table}
\end{exemple}

%\end{document}
