%utiliser les environnement \begin{comment} \end{comment} pour mettre en commentaire le préambule une fois la programmation appelée dans le document maître (!ne pas oublier de mettre en commentaire \end{document}!)

%\begin{comment}

\documentclass[a4paper, 11pt, twoside, fleqn]{memoir}

\usepackage{AOCDTF}

%--------------------------------------
%CANEVAS
%--------------------------------------

\newcommand\BoxColor{\ifcase\thechapshift blue!30\or brown!30\or pink!30\or cyan!30\or green!30\or teal!30\or purple!30\or red!30\or olive!30\or orange!30\or lime!30\or gray!\or magenta!30\else yellow!30\fi} %définition de la couleur des marqueurs de chapitre

\newcounter{chapshift} %compteur de chapitre du marqueur de chapitre
\addtocounter{chapshift}{-1}
	
\newif\ifFrame %instruction conditionnelle pour les couleurs des pages
\Frametrue

\pagestyle{plain}

% the main command; the mandatory argument sets the color of the vertical box
\newcommand\ChapFrame{%
\AddEverypageHook{%
\ifFrame
\ifthenelse{\isodd{\value{page}}}
  {\backgroundsetup{contents={%
  \begin{tikzpicture}[overlay,remember picture]
  \node[
  	rounded corners=3pt,
    fill=\BoxColor,
    inner sep=0pt,
    rectangle,
    text width=1.3cm,
    text height=5.5cm,
    align=center,
    anchor=north west
  ] 
  at ($ (current page.north west) + (-0cm,-2*\thechapshift cm) $) %nombre négatif = espacement des marqueurs entre les différents chapitres (à régler en fin de rédaction) (4.5cm vaut un espacement équivalement à la hauteur du marqueur, une page peut en contenir 6 avec cet espacement-la mais il est le plus équilibré)
    {\rotatebox{90}{\hspace*{.3cm}%
      \parbox[c][1.2cm][t]{5cm}{%
        \raggedright\textcolor{black}{\sffamily\textbf{\leftmark}}}}};
  \end{tikzpicture}}}
  }
  {\backgroundsetup{contents={%
  \begin{tikzpicture}[overlay,remember picture]
  \node[
  	rounded corners=3pt,
    fill=\BoxColor,
    inner sep=0pt,
    rectangle,
    text width=1.3cm,
    text height=5.5cm,
    align=center,
    anchor=north east
  ] 
  at ($ (current page.north east) + (-0cm,-2*\thechapshift cm) $) %nombre négatif = espacement des marqueurs entre les différents chapitres (à régler en fin de rédaction) (4.5cm vaut un espacement équivalement à la hauteur du marqueur, une page peut en contenir 6 avec cet espacement-la mais il est le plus équilibré)
    {\rotatebox{90}{\hspace*{.3cm}%
      \parbox[c][1.2cm][t]{5cm}{%
        \raggedright\textcolor{black}{\sffamily\textbf{\leftmark}}}}};
  \end{tikzpicture}}}%
  }
  \BgMaterial%
  \fi%
}%
  \stepcounter{chapshift}
}

\renewcommand\chaptermark[1]{\markboth{\thechapter.~#1}{}} %redéfinition du marqueur de chapitre pour ne contenir que le titre du chapitre %à personnaliser selon le nombre de chapitre dans le cours

%--------------------------------------
%corps du document
%--------------------------------------

\begin{document} %corps du document
	\openleft %début de chapitre à gauche

%\end{comment}

\begin{xltabular}{\textwidth}{X p{5.5cm} p{7.5cm}} %tableau sur plusieurs pages avec des colonnes contenant des \item donc la largeur doit être déclarée
	\caption{Description des nombres quantiques \label{tab:description_nombres_quantiques}} \\
	\toprule %filet de milieu de tableau
	\thead{Nombre\\quantique} & \thead{Valeurs} & \thead{Description} \\
	\midrule %filet de milieu de tableau
\endfirsthead %en-tête de la première page du tableau  

	\caption{Description des nombres quantiques} \\
	\toprule 
	\thead{Nombre\\quantique} & \thead{Valeurs} & \thead{Description}\\
	\midrule
\endhead %en-tête de toutes les pages du tableau

	\midrule %filet de milieu de tableau
	\multicolumn{3}{r}{\small\textit{Page suivante}}
\endfoot %pied de page de toutes les pages du tableau
\endlastfoot %pied de page de la dernièredu tableau

\multicolumn{3}{l}{\textit{Nombre quantique principal $n$}} \\ 
\middashrule %filet de milieu de tableau pointillé

&
\begin{tabdescription} %appel d'une liste de description pour le tableau
 	\item[$n\ge1$ :] Entier positif non nul
 	\item[Exemple :]\hfill %suppression de la première ligne
		\begin{compactitemize} %appel d'une liste compacte
 			\item $n=1$\,;
			\item $n=2$\,;
 			\item $n=3$\,;
			\item ...
		\end{compactitemize}
\end{tabdescription} 
&
\begin{tabdescription}
	\item[Définition de la couche électronique :] distance entre le noyau et l'électron.  
	\item[Exemple :]\hfill
		\begin{compactitemize}
			\item $n=1$ pour la couche K\,;
 			\item $n=2$ pour la couche L\,;
 			\item $n=3$ pour la couche M\,;
 			\item ...
		\end{compactitemize}
\end{tabdescription} \\ 

\addlinespace %ajout d'un espace avant la partie suivante
\multicolumn{3}{l}{\textit{Nombre quantique secondaire/azumital $\ell$}} \\
\middashrule %filet de milieu de tableau pointillé
& 
\begin{tabdescription}
	\item[$0\ge \ell<n-1$ :] Entier positif à $n$ valeur(s)
 	\item[Exemple :]\hfill
 		\begin{compactitemize}
 			\item $\ell=0$\,;
			\item $\ell=1$\,;
 			\item $\ell=2$\,;
 			\item Jusqu'à $\ell=(n-1)$\,.
		\end{compactitemize}
\end{tabdescription} 
&
\begin{tabdescription}
	\item[Définition de la sous-couche électronique :] forme et symétrie de l'orbitale atomique.  
	\item[Valeurs :]\hfill
		\begin{compactitemize}
			\item $\ell=0$ pour la sous-couche s (\textbf{s}harp)\,;
			\item $\ell=1$ pour la sous-couche p (\textbf{p}rincipal)\,;
			\item $\ell=2$ pour la sous-couche d (\textbf{d}iffuse)\,;
			\item $\ell=3$ pour la sous-couche f  (\textbf{f}ondamental).
		\end{compactitemize}
	\item[Forme :]\hfill
		\begin{tabdescription}
			\item[$\ell=0$ :] 1 lobe\,;
			\item[$\ell=1$ :] 2 lobes\,;
			\item[$\ell=2$ :] 4 lobes\,;
			\item[$\ell=3$ :] 8 lobes.
		\end{tabdescription}
\end{tabdescription} \\

%\noalign{\break} %impose le saut de page au tableau tout en répartissant verticalement le tableau

\addlinespace %ajout d'un espace avant la partie suivante
\multicolumn{3}{l}{\textit{Nombre quantique tertiaire/magnétique $m_\ell$}} \\ 
\middashrule %filet de milieu de tableau pointillé
&
\begin{tabdescription}
	\item[$-\ell\ge m_l<+l$ :] Entier positif à $(2\ell+n)$ valeur(s)
 	\item[Exemple :]\hfill
 		\begin{compactitemize}
			\item $-\ell$\,;
 			\item $(-\ell+1)$\,;
			\item $(-\ell+2)$\,;
 			\item ...
 			\item $0$\,;
 			\item ...
 			\item $(\ell-2)$\,;
 			\item $(\ell-1)$\,;
 			\item $+\ell$.
		\end{compactitemize}
\end{tabdescription} 
& 
\begin{tabdescription}
	\item[Définition de l'orientation :] orientation de l'orbitale dans l'espaces selon les axes.  
	\item[Exemple si $\ell=2$ :]\hfill
		\begin{compactitemize}
 			\item Forme d'haltères croisées;
 			\item $m_\ell=$ \numlist[list-separator = {; }, list-final-separator = {; }]{-2; -1;0;1;2}. %liste de nombre
		\end{compactitemize}
\end{tabdescription} \\ 

\addlinespace %ajout d'un espace avant la partie suivante
\multicolumn{3}{l}{\textit{Nombre quantique du spin $S$}} \\ 
\middashrule %filet de milieu de tableau pointillé

& 
$S=1/2$
& 
Moment magnétique dû à la rotation de l'électron sur lui-même.\\ 

\addlinespace %ajout d'un espace avant la partie suivante
\multicolumn{3}{l}{\textit{Nombre quantique magnétique du spin $m_S$}} \\ 
\middashrule %filet de milieu de tableau pointillé
& 
$m_S=$ \numlist[list-pair-separator = {; }]{-1/2;1/2}
& 
Sens de rotation de l'électron sur lui-même.\\ 

\bottomrule %filet de fin de tableau
\end{xltabular}

\end{document}