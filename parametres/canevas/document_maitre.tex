%--------------------------------------
%CANEVAS
%--------------------------------------

%--------------------------------------
%appel de la classe de document et de ses options
%--------------------------------------

\documentclass[a4paper, 11pt, twoside, fleqn]{memoir}

\usepackage{AOCDTF}

%--------------------------------------
%données du document
%--------------------------------------

\title{Matière -- Cours}
\date{\year}
\author[1]{Prénom1 Nom1}
%\author[2]{Prénom2 Nom2}
%\author[3]{Prénom3 Nom3}
%\author[4]{Prénom4 Nom4}

%--------------------------------------
%programmations spécifiques de mise-en-pages adaptables selon les matières et bibliographie
%--------------------------------------

%--------------------------------------
%CANEVAS
%--------------------------------------

\newcommand\BoxColor{\ifcase\thechapshift blue!30\or brown!30\or pink!30\or cyan!30\or green!30\or teal!30\or purple!30\or red!30\or olive!30\or orange!30\or lime!30\or gray!\or magenta!30\else yellow!30\fi} %définition de la couleur des marqueurs de chapitre

\newcounter{chapshift} %compteur de chapitre du marqueur de chapitre
\addtocounter{chapshift}{-1}
	
\newif\ifFrame %instruction conditionnelle pour les couleurs des pages
\Frametrue

\pagestyle{plain}

% the main command; the mandatory argument sets the color of the vertical box
\newcommand\ChapFrame{%
\AddEverypageHook{%
\ifFrame
\ifthenelse{\isodd{\value{page}}}
  {\backgroundsetup{contents={%
  \begin{tikzpicture}[overlay,remember picture]
  \node[
  	rounded corners=3pt,
    fill=\BoxColor,
    inner sep=0pt,
    rectangle,
    text width=1.3cm,
    text height=5.5cm,
    align=center,
    anchor=north west
  ] 
  at ($ (current page.north west) + (-0cm,-2*\thechapshift cm) $) %nombre négatif = espacement des marqueurs entre les différents chapitres (à régler en fin de rédaction) (4.5cm vaut un espacement équivalement à la hauteur du marqueur, une page peut en contenir 6 avec cet espacement-la mais il est le plus équilibré)
    {\rotatebox{90}{\hspace*{.3cm}%
      \parbox[c][1.2cm][t]{5cm}{%
        \raggedright\textcolor{black}{\sffamily\textbf{\leftmark}}}}};
  \end{tikzpicture}}}
  }
  {\backgroundsetup{contents={%
  \begin{tikzpicture}[overlay,remember picture]
  \node[
  	rounded corners=3pt,
    fill=\BoxColor,
    inner sep=0pt,
    rectangle,
    text width=1.3cm,
    text height=5.5cm,
    align=center,
    anchor=north east
  ] 
  at ($ (current page.north east) + (-0cm,-2*\thechapshift cm) $) %nombre négatif = espacement des marqueurs entre les différents chapitres (à régler en fin de rédaction) (4.5cm vaut un espacement équivalement à la hauteur du marqueur, une page peut en contenir 6 avec cet espacement-la mais il est le plus équilibré)
    {\rotatebox{90}{\hspace*{.3cm}%
      \parbox[c][1.2cm][t]{5cm}{%
        \raggedright\textcolor{black}{\sffamily\textbf{\leftmark}}}}};
  \end{tikzpicture}}}%
  }
  \BgMaterial%
  \fi%
}%
  \stepcounter{chapshift}
}

\renewcommand\chaptermark[1]{\markboth{\thechapter.~#1}{}} %redéfinition du marqueur de chapitre pour ne contenir que le titre du chapitre %à personnaliser selon le nombre de chapitre dans le cours

\addbibresource{../bibliographies/bib_chimie.bib} %à redéfinir une fois le cours entamé le chemin ne sera pas le même selon les cours !

%--------------------------------------
%corps du document
%--------------------------------------

\begin{document} %corps du document

%--------------------------------------
%préface, page de couverture, table des matières...
%--------------------------------------

\frontmatter
	
\Framefalse %défini la booléenne Frame comme faux
	
	%--------------------------------------
	%page de couverture et de titre
	%--------------------------------------

	\input{page_couverture}

	%--------------------------------------
%PRE-REQUIS
%--------------------------------------

%utiliser les environnement \begin{comment} \end{comment} pour mettre en commentaire le préambule une fois la programmation appelée dans le document maître (!ne pas oublier de mettre en commentaire \end{document}!)

\begin{comment}

\documentclass[a4paper, 11pt, twoside, fleqn]{memoir}

\usepackage{AOCDTF}

%--------------------------------------
%CANEVAS
%--------------------------------------

\newcommand\BoxColor{\ifcase\thechapshift blue!30\or brown!30\or pink!30\or cyan!30\or green!30\or teal!30\or purple!30\or red!30\or olive!30\or orange!30\or lime!30\or gray!\or magenta!30\else yellow!30\fi} %définition de la couleur des marqueurs de chapitre

\newcounter{chapshift} %compteur de chapitre du marqueur de chapitre
\addtocounter{chapshift}{-1}
	
\newif\ifFrame %instruction conditionnelle pour les couleurs des pages
\Frametrue

\pagestyle{plain}

% the main command; the mandatory argument sets the color of the vertical box
\newcommand\ChapFrame{%
\AddEverypageHook{%
\ifFrame
\ifthenelse{\isodd{\value{page}}}
  {\backgroundsetup{contents={%
  \begin{tikzpicture}[overlay,remember picture]
  \node[
  	rounded corners=3pt,
    fill=\BoxColor,
    inner sep=0pt,
    rectangle,
    text width=1.3cm,
    text height=5.5cm,
    align=center,
    anchor=north west
  ] 
  at ($ (current page.north west) + (-0cm,-2*\thechapshift cm) $) %nombre négatif = espacement des marqueurs entre les différents chapitres (à régler en fin de rédaction) (4.5cm vaut un espacement équivalement à la hauteur du marqueur, une page peut en contenir 6 avec cet espacement-la mais il est le plus équilibré)
    {\rotatebox{90}{\hspace*{.3cm}%
      \parbox[c][1.2cm][t]{5cm}{%
        \raggedright\textcolor{black}{\sffamily\textbf{\leftmark}}}}};
  \end{tikzpicture}}}
  }
  {\backgroundsetup{contents={%
  \begin{tikzpicture}[overlay,remember picture]
  \node[
  	rounded corners=3pt,
    fill=\BoxColor,
    inner sep=0pt,
    rectangle,
    text width=1.3cm,
    text height=5.5cm,
    align=center,
    anchor=north east
  ] 
  at ($ (current page.north east) + (-0cm,-2*\thechapshift cm) $) %nombre négatif = espacement des marqueurs entre les différents chapitres (à régler en fin de rédaction) (4.5cm vaut un espacement équivalement à la hauteur du marqueur, une page peut en contenir 6 avec cet espacement-la mais il est le plus équilibré)
    {\rotatebox{90}{\hspace*{.3cm}%
      \parbox[c][1.2cm][t]{5cm}{%
        \raggedright\textcolor{black}{\sffamily\textbf{\leftmark}}}}};
  \end{tikzpicture}}}%
  }
  \BgMaterial%
  \fi%
}%
  \stepcounter{chapshift}
}

\renewcommand\chaptermark[1]{\markboth{\thechapter.~#1}{}} %redéfinition du marqueur de chapitre pour ne contenir que le titre du chapitre %à personnaliser selon le nombre de chapitre dans le cours

%--------------------------------------
%corps du document
%--------------------------------------

\begin{document} %corps du document
	\openleft %début de chapitre à gauche

\end{comment}

\pagestyle{empty}
\thispagestyle{empty}

\begingroup

%--------------------------------------
%logo Compagnon
%--------------------------------------
 
\begin{tikzpicture}[remember picture, overlay]
	\begin{scope}[shift={(current page.north west)}]
		\draw (2.5,-1)
		node [below right]{\includegraphics[scale=0.1]{logo_compagnons_nom.png}};
	\end{scope}
\end{tikzpicture}

~\\[5cm] %espace insécable pour marquer le début du texte sur l'environnement et situer d'autres éléments de texte sur la page

%--------------------------------------
%titre de la page
%--------------------------------------

\begin{flushleft}
	\HUGE\sffamily\textbf{Physique -- chimie}\\
\end{flushleft}
\HRule
\begin{flushright}
	\huge\sffamily\textbf{Pré-requis}\\[0.4cm]
\end{flushright}
~\\[2cm]

%--------------------------------------
%auteur et logo MTA
%--------------------------------------

\begin{minipage}{0.3\textwidth}
	\begin{flushleft} \large
		\rmfamily Bruno \textsc{Douchy}
		%\\
		%\rmfamily Prénom2 \textsc{Nom2}
	\end{flushleft}
\end{minipage}
\hfill
\begin{minipage}{0.3\textwidth}
	\centering
	\includegraphics[scale=0.25]{logo_metier.png}
\end{minipage}

\begin{comment}
\hfill
\begin{minipage}{0.3\textwidth}
	\begin{flushright} \large
		\rmfamily Prénom3 \textsc{Nom3} %auteur supplémentaires à rajouter en mettant en commentaire l'environnement comment
		\\
 		\rmfamily Prénom4 \textsc{Nom4}
	\end{flushright}
\end{minipage}
\end{comment}

\vfill
\begin{flushleft}
	\sffamily\bfseries\large\ Édition \the\year.\the\month
	\\
	~v\version
\end{flushleft}
\vfill

%--------------------------------------
%pied de page Compagnon
%--------------------------------------

\begin{tikzpicture}[remember picture, overlay]
	\begin{scope}[shift={(current page.south west)}]
		\draw (2.5,1)
		node [above right]{\includegraphics{pied_page_compagnons.png}};
	\end{scope}
\end{tikzpicture}

\endgroup
\null\clearpage

%\end{document}

	\pagestyle{plain} %style de page avec en-tête et pied-de-page
	\pagenumbering{roman}
	\openany
	
	%--------------------------------------
	%listes de contenus
	%--------------------------------------
	
	{\hypersetup{linkcolor=black}\tableofcontents} %table des matières en noir
	\newpage
	{\hypersetup{linkcolor=black}\listoftables} %liste des tableaux en noir
	\newpage
	{\hypersetup{linkcolor=black}\listoffigures} %liste des figures en noir
	{\hypersetup{linkcolor=black}\listtheoremname\listtheorems{Theorem}} %liste des équations en noir
	{\hypersetup{linkcolor=black}\listdefinitionname\listtheorems{Definition}} %liste des définitions en noir
		
	\openright %début de chapitre à "droite" mais comme demarrage de la numérotation inversé avec la page de titre, ça décale l'ouverture des chapitre à gauche

	%--------------------------------------
	%chapitre d'introduction
	%--------------------------------------

	\input{chap_0_preface}
		
%--------------------------------------
%corps de texte, annexes
%--------------------------------------

\mainmatter

\Frametrue %défini la booléenne Frame comme vrai -> marqueurs de chapitre

	%--------------------------------------
	%inclusion des chapitres
	%--------------------------------------

\input{chap_1_premier_chapitre}

\input{chap_2_deuxieme_chapitre}

\input{chap_3_troisieme_chapitre}

	%--------------------------------------
	%style des annexes
	%--------------------------------------

	\Framefalse %défini la booléenne Frame comme false -> pas de marqueurs de chapitre
	\appendix %appel des annexes
	\appendixpage

	%--------------------------------------
	%inclusion des chapitres
	%--------------------------------------

	\input{chap_A_premiere_annexe}
	
	\input{chap_B_deuxieme_annexe}


%--------------------------------------
%conclusion, bibliographie
%--------------------------------------

\backmatter

	%--------------------------------------
	%inclusion des chapitres
	%--------------------------------------

	\input{chap_conclusion}

	%--------------------------------------
	%bibliographie
	%--------------------------------------

	\nocite{Coarer2003}
	
	\printbibliography %ajout des références bibliographiques
		
\end{document}

