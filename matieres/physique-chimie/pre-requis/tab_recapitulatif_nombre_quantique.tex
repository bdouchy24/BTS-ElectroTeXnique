%utiliser les environnement \begin{comment} \end{comment} pour mettre en commentaire le préambule une fois la programmation appelée dans le document maître (!ne pas oublier de mettre en commentaire \end{document}!)

%\begin{comment}

\documentclass[a4paper, 11pt, twoside, fleqn]{memoir}

\usepackage{AOCDTF}

%--------------------------------------
%CANEVAS
%--------------------------------------

\newcommand\BoxColor{\ifcase\thechapshift blue!30\or brown!30\or pink!30\or cyan!30\or green!30\or teal!30\or purple!30\or red!30\or olive!30\or orange!30\or lime!30\or gray!\or magenta!30\else yellow!30\fi} %définition de la couleur des marqueurs de chapitre

\newcounter{chapshift} %compteur de chapitre du marqueur de chapitre
\addtocounter{chapshift}{-1}
	
\newif\ifFrame %instruction conditionnelle pour les couleurs des pages
\Frametrue

\pagestyle{plain}

% the main command; the mandatory argument sets the color of the vertical box
\newcommand\ChapFrame{%
\AddEverypageHook{%
\ifFrame
\ifthenelse{\isodd{\value{page}}}
  {\backgroundsetup{contents={%
  \begin{tikzpicture}[overlay,remember picture]
  \node[
  	rounded corners=3pt,
    fill=\BoxColor,
    inner sep=0pt,
    rectangle,
    text width=1.5cm,
    text height=5.5cm,
    align=center,
    anchor=north west
  ] 
  at ($ (current page.north west) + (-0cm,-2*\thechapshift cm) $) %nombre négatif = espacement des marqueurs entre les différents chapitres (à régler en fin de rédaction) (4.5cm vaut un espacement équivalement à la hauteur du marqueur, une page peut en contenir 6 avec cet espacement-la mais il est le plus équilibré)
    {\rotatebox{90}{\hspace*{.5cm}%
      \parbox[c][1.2cm][t]{5cm}{%
        \raggedright\textcolor{black}{\sffamily\textbf{\leftmark}}}}};
  \end{tikzpicture}}}
  }
  {\backgroundsetup{contents={%
  \begin{tikzpicture}[overlay,remember picture]
  \node[
  	rounded corners=3pt,
    fill=\BoxColor,
    inner sep=0pt,
    rectangle,
    text width=1.5cm,
    text height=5.5cm,
    align=center,
    anchor=north east
  ] 
  at ($ (current page.north east) + (-0cm,-2*\thechapshift cm) $) %nombre négatif = espacement des marqueurs entre les différents chapitres (à régler en fin de rédaction) (4.5cm vaut un espacement équivalement à la hauteur du marqueur, une page peut en contenir 6 avec cet espacement-la mais il est le plus équilibré)
    {\rotatebox{90}{\hspace*{.5cm}%
      \parbox[c][1.2cm][t]{5cm}{%
        \raggedright\textcolor{black}{\sffamily\textbf{\leftmark}}}}};
  \end{tikzpicture}}}%
  }
  \BgMaterial%
  \fi%
}%
  \stepcounter{chapshift}
}

\renewcommand\chaptermark[1]{\markboth{\thechapter.~#1}{}} %redéfinition du marqueur de chapitre pour ne contenir que le titre du chapitre %à personnaliser selon le nombre de chapitre dans le cours

%--------------------------------------
%corps du document
%--------------------------------------

\begin{document} %corps du document
	\openleft %début de chapitre à gauche

%\end{comment}

\begin{longtable}{c r r r c p{7,2cm}} \\ %tableau de plusieurs à 6 colonnes, si usage itemize, indiquer la lagueur de la colonne dans les appels des colonnes
	\caption{Récapitulatif des trois premières couches électroniques\label{tab:recap_nombre_quantique}} \\
	\toprule 
	\thead{Couche} & \thead[r]{$n$} & \thead[r]{$l$} & \thead[r]{$m_l$} & \thead{Cases\\quantiques} & \thead{Forme orbitale} \\
	\midrule 
\endfirsthead %en-tête de la première page du tableau  

	\toprule 
	\thead{Couche} & \thead{$n$} & \thead{$l$} & \thead{$m_l$} & \thead{Cases\\quantiques} & \thead{Forme orbitale} \\
	\midrule 	
	\multicolumn{6}{l}{\textit{Couche L}} \\
\middashrule  %filet de milieu de tableau pointillé
\endhead %en-tête de toutes les pages du tableau

	\midrule %filet de milieu de tableau
	\multicolumn{3}{r}{\small\textit{Page suivante}}
\endfoot %pied de page de toutes les pages du tableau
\endlastfoot %pied de page de la dernièredu tableau

\multicolumn{6}{l}{\textit{Couche K}} \\
\middashrule %filet de milieu de tableau pointillé

& \multirow[t]{2}{*}{$1$} & \multirow[t]{2}{*}{$0$} & \multirow[t]{2}{*}{$0$} & %multiligne pour séparer l'image du texte de la dernière colonne
\multirow[t]{2}{*}{
\adjustbox{valign=t}{ %alignement des figures avec le haut de la cellule
	\begin{MOdiagram}[style=square, labels]
		\atom{left}{1s}
	\end{MOdiagram}}}
&
\begin{tabdescription}
	\item[Sphère :]\hfill
	\begin{compactitemize}
		\item 1 seule orientation
		\item Représentation :
	\end{compactitemize}
\end{tabdescription} \\

& & & & & %multiligne pour séparer l'image du texte
\begin{center}
	\begin{tikzpicture}
		\orbital[pos = {(0,5.5)}]{s}
		\node[above] at (0,6) {s};
	\end{tikzpicture}
\end{center}\\ 

\multicolumn{6}{l}{\textit{Couche L}} \\
\middashrule %filet de milieu de tableau pointillé

& \multirow[t]{2}{*}{$2$} & \multirow[t]{2}{*}{$0$} & \multirow[t]{2}{*}{$0$} & %multiligne pour séparer l'image du texte de la dernière colonne
\multirow[t]{2}{*}{
\adjustbox{valign=t}{ %alignement des figures avec le haut de la cellule
	\begin{MOdiagram}[style=square, labels]
		\atom{left}{2s}
	\end{MOdiagram}}}
&
\begin{tabdescription}
	\item[Sphère :]\hfill
	\begin{compactitemize}
		\item 1 seule orientation
		\item Représentation :
	\end{compactitemize}
\end{tabdescription} \\

& & & & & %multiligne pour séparer l'image du texte
\begin{center}
	\begin{tikzpicture}
		\orbital[pos = {(0,5.5)}]{s}
		\node[above] at (0,6) {s};
	\end{tikzpicture}
\end{center}\\ 

& \multirow[t]{4}{*}{$2$} & \multirow[t]{4}{*}{$1$} & $-1$ & %multiligne pour séparer l'image du texte de la dernière colonne
\multirow[t]{4}{*}{
\adjustbox{valign=t}{ %alignement des figures avec le haut de la cellule
\begin{MOdiagram}[style=square, labels]
	\atom[2p]{left}{2p}
\end{MOdiagram}}}
&
\multirow[t]{3}{7,2cm}{
\begin{tabdescription}
	\item[\og Haltères \fg{} :]\hfill
		\begin{compactitemize}
			\item 3 orientations dans l'espace $P_x; P_y; P_z$
			\item Représentations :
		\end{compactitemize}
\end{tabdescription}}\\

& & & $0$ & & \\

& & & +$1$ & & 
\begin{center}
	\begin{tikzpicture}
		\orbital[pos = {(0,3)}]{px}
		\node[above] at (0,4) {p$_x$};
		\orbital[pos = {(2,3)}]{py}
		\node[above] at (2,4) {p$_y$};
		\orbital[pos = {(4,3)}]{pz}
		\node[above] at (4,4) {p$_z$};
	\end{tikzpicture}
\end{center}\\

\multicolumn{6}{l}{\textit{Couche M}} \\
\middashrule %filet de milieu de tableau pointillé

& \multirow[t]{12}{*}{$3$} & \multirow[t]{2}{*}{$0$} & \multirow[t]{2}{*}{$0$} & %multiligne pour séparer l'image du texte de la dernière colonne
\multirow[t]{2}{*}{
\adjustbox{valign=t}{ %alignement des figures avec le haut de la cellule
	\begin{MOdiagram}[style=square, labels]
     	\AO(0cm){s}[label={$3s$}]{0}
	\end{MOdiagram}}}
&
\begin{tabdescription}
	\item[Sphère :]\hfill
	\begin{compactitemize}
		\item 1 seule orientation
		\item Représentation :
	\end{compactitemize}
\end{tabdescription} \\

& & & & & %multiligne pour séparer l'image du texte dans la dernière colonne
\begin{center}
	\begin{tikzpicture}
		\orbital[pos = {(0,5.5)}]{s}
		\node[above] at (0,6) {s};
	\end{tikzpicture}
\end{center}\\ 

& & \multirow[t]{4}{*}{$1$} & $-1$ & %multiligne pour séparer l'image du texte de la dernière colonne
\multirow[t]{4}{*}{
\adjustbox{valign=t}{ %alignement des figures avec le haut de la cellule
	\begin{MOdiagram}[style=square, labels]
        \AO(0cm){s}[label={$3p_{x}$}]{0}
        \AO(0,7cm){s}[label={$3p_{y}$}]{0}
        \AO(1,4cm){s}[label={$3p_{z}$}]{0}
	\end{MOdiagram}}}
&
\multirow[t]{3}{7,2cm}{
\begin{tabdescription}
	\item[\og Haltères \fg{} :]\hfill
		\begin{compactitemize}
			\item 3 orientations dans l'espace $P_x; P_y; P_z$
			\item Représentations :
		\end{compactitemize}
\end{tabdescription}}\\

& & & $0$ & & \\

& & & $+1$ & & 
\begin{center}
	\begin{tikzpicture}
		\orbital[pos = {(0,3)}]{px}
		\node[above] at (0,4) {p$_x$};
		\orbital[pos = {(2,3)}]{py}
		\node[above] at (2,4) {p$_y$};
		\orbital[pos = {(4,3)}]{pz}
		\node[above] at (4,4) {p$_z$};
	\end{tikzpicture}
\end{center}\\

& & \multirow[t]{6}{*}{$2$} & $-2$ & %multiligne pour séparer l'image du texte de la dernière colonne
\multirow[t]{6}{*}{
\adjustbox{valign=t}{ %alignement des figures avec le haut de la cellule
\begin{MOdiagram}[style=square, labels]
        \AO(0cm){s}[label={$3d_{xz}$}]{0}
        \AO(1cm){s}[label={$3d_{x^2}$}]{0}
        \AO(2cm){s}[label={$3d_{x^2-z^2}$}]{0}
        \AO(3cm){s}[label={$3d_{xy}$}]{0}
        \AO(4cm){s}[label={$3d_{yz}$}]{0}
\end{MOdiagram}}}
&
\multirow[t]{5}{7,2cm}{
\begin{tabdescription}
	\item[\og Haltères croisées \fg{} :]\hfill
		\begin{compactitemize}
			\item 5 orientations dans l'espace $d_xz; d_{x^2}; d_{x^2-z^2}; d_{xy}; d_{yz}$
			\item Représentations :
		\end{compactitemize}
\end{tabdescription}} \\

& & & $-1$ & & \\

& & & $0$ & & \\

& & & $+1$ & & \\

& & & $+2$ & &
\begin{center}
	\begin{tikzpicture}
		\orbital[pos = {(0,-3)}]{dxy}
		\node[above] at (0,-2.1) {d$_{xy}$};
		\orbital[pos = {(2,-3)}]{dxz}
		\node[above] at (2,-2.1) {d$_{xz}$};
		\orbital[pos = {(4,-3)}]{dyz}
		\node[above] at (4,-2) {d$_{yz}$};
		\orbital[pos = {(0,-5)}]{dx2y2}
		\node[below] at (0,-5.9) {d$_{x^2-y^2}$};
		\orbital[pos = {(2,-5)}]{dz2}
		\node[below] at (2,-5.9) {d$_{z^2}$};
	\end{tikzpicture}
\end{center} \\

\bottomrule
\end{longtable}

\end{document}