%--------------------------------------
%PRE-REQUIS
%--------------------------------------

%utiliser les environnement \begin{comment} \end{comment} pour mettre en commentaire le préambule une fois la programmation appelée dans le document maître (!ne pas oublier de mettre en commentaire \end{document}!)

%\begin{comment}

\documentclass[a4paper, 11pt, twoside, fleqn]{memoir}

%--------------------------------------
%préambule
%--------------------------------------

%--------------------------------------
%PREAMBULE COMMUN A TOUS LES COURS - A MODIFIER AVEC PRUDENCE
%--------------------------------------

	%--------------------------------------
	%Packages pour le document
	%--------------------------------------

	\usepackage[french]{babel}
	\usepackage{lmodern, marvosym, microtype} %gestion fine des police de caractère
	\usepackage[table, svgnames, dvipsnames, x11names]{xcolor} %gestion des couleurs
	\usepackage[bookmarksnumbered=false, breaklinks, linktoc=all]{hyperref} %référencage
	\usepackage{varioref, memhfixc, url} % amélioration du référencage
	\usepackage{pict2e, picture, multicol, pdflscape, graphicx, eso-pic, preview, graphbox, wrapfig}
	\usepackage{callouts, circledsteps} %annotation d'image
	\usepackage[absolute]{textpos}  % disposition d'images
	\usepackage{authblk, tocbibind, calc} %gestion mise en page
	\usepackage{xifthen, multido, etoolbox, ifpdf} %appel de fonction conditionnelles
	\usepackage{mathtools, amsfonts, amssymb, mathrsfs} %écriture des mathématiques avec référénces
	\usepackage{chemfig, bohr, tikzorbital, chemgreek, expl3, xparse, l3keys2e, xargs, verbatim} %gestion de l'écriture en chimie
	\usepackage{modiagram} %orbitale atomique
	\usepackage{cancel, colortbl, csquotes, mathpazo, soul, caption} %packages nécessaires pour le chargement du package SIunitx
	\usepackage{siunitx} %gestion des unités de physique
	\usepackage[export]{adjustbox}
	\usepackage{animate} %animation d'image
	\usepackage{subcaption, capt-of} %gestion des sous-figures et des légendes communes
	\usepackage{enumitem} %mise en page type code informatique et listes
	\usepackage{verbatim} %mise en page type code informatique et listes
	\usepackage{rotating, mdframed} %rotation
	\usepackage[scale=1,angle=0,opacity=1,contents={}]{background} %gestion de l'arrière-plan 
	\usepackage{xspace, xpatch} %espace après les macros
	\usepackage{pgf, tikz, tikz-qtree, pgfplots, pgfplotstable, schemabloc} %création de figures et schémas
	\usepackage{smartdiagram}
	\usesmartdiagramlibrary{additions}
	\usepackage[siunitx, europeanvoltages, europeancurrents, europeanresistors, americaninductors, europeanports, europeangfsurgearrester, compatibility]{circuitikz}%création de schémas électriques
	\usepackage{pdfpages, scrbase} %inclusion des PDF
	\usepackage{xstring, multirow, booktabs, longtable, makecell, arydshln, cellspace, tabu} %gestion fine des tableaux (sans \ltablex !)
	\usepackage{threeparttable} %notes dans tableaux
	\usepackage[section]{placeins} %force les figures à rester dans leur section
	\usepackage{impnattypo} %règle de typographie française
	\let\newfloat\undefined\usepackage{floatrow} %nouvel environnement flottant

		%--------------------------------------
		%packages aidant à la rédaction
		%--------------------------------------

		\usepackage{lipsum} %insertion LIPSUM
		%\usepackage{showframe} %montre la structure du document
		\usepackage{comment} %commentaire de code sur plusieurs lignes

	%--------------------------------------
	%Packages pour la bibliographie
	%--------------------------------------

	\usepackage[backend=biber, style=numeric, hyperref=auto, citestyle=numeric-comp, autopunct=false]{biblatex}
	\usepackage{biblatex-anonymous}
	\usepackage{csquotes}
	
	%--------------------------------------
	%Pramètrage de la bibliographie
	%--------------------------------------
	
	%\DefineBibliographyStrings{french}{in={dans},inseries={dans}}

	\NewBibliographyString{chapitre}
	\DefineBibliographyStrings{french}{chapitre = {Chap.},}

%--------------------------------------
%paramétrage des packages pour le document
%--------------------------------------

\hypersetup{colorlinks = true, urlcolor = Blue, linkcolor = ForestGreen, citecolor = Red} %paramètrage des couleurs des liens

%\frenchbsetup{StandardLists=true} %liste au format français

\sisetup{%
	locale=FR, %règles de typo française
	detect-all, %on prend la font du document
	group-digits=integer, %le regroupement par 3 chiffres n’a lieu qu’en partie gauche
	free-standing-units, %macro pour les unités existants en dehors des arguments \si et \SI
	group-minimum-digits=5, %groupe si au moins 5 chiffres
	load-configurations=abbreviations % charge les abréviations avec l'argument \SI
	}

	%--------------------------------------
	%paramétrage Tikz-PGF
	%--------------------------------------

	\usetikzlibrary{arrows.meta, pgfplots.dateplot, arrows, shapes.misc, positioning, mindmap, plotmarks, shapes.callouts, fit, matrix, intersections, decorations, decorations.pathmorphing, decorations.pathreplacing, decorations.shapes, decorations.text, decorations.markings, decorations.fractals, decorations.footprints} 
		
	\tikzset{mynode/.style={draw=black, solid, circle, fill=white, inner sep=2pt, thick, text=black}} %pastille d'annotation
	
	\usesmartdiagramlibrary{additions}

	\usepgfplotslibrary{dateplot} %insertion de dates en coordonnées dans les graphiques
	\pgfplotsset{compat=1.16}
	
	\tikzstyle{element}=[rectangle, draw, minimum width=3cm, text centered, text width=2.5cm] %bloc de texte utilisé dans les schémas fléchés

		%--------------------------------------
		%grille d'aide au placement des éléments
		%--------------------------------------
	
		\makeatletter
		\newif\ifmygrid@coordinates
			\tikzset{/mygrid/step line/.style={line width=0.80pt,draw=gray!80},
         		/mygrid/steplet line/.style={line width=0.25pt,draw=gray!80}}
			\pgfkeys{/mygrid/.cd,
         		step/.store in=\mygrid@step,
        			steplet/.store in=\mygrid@steplet,
         		coordinates/.is if=mygrid@coordinates}
			\def\mygrid@def@coordinates(#1,#2)(#3,#4){%
   				\def\mygrid@xlo{#1}%
    				\def\mygrid@xhi{#3}%
    				\def\mygrid@ylo{#2}%
    				\def\mygrid@yhi{#4}%
			}
		\newcommand\DrawGrid[3][]{%
    			\pgfkeys{/mygrid/.cd,coordinates=true,step=1,steplet=0.2,#1}%
    			\draw[/mygrid/steplet line] #2 grid[step=\mygrid@steplet] #3;
    			\draw[/mygrid/step line] #2 grid[step=\mygrid@step] #3;
    			\mygrid@def@coordinates#2#3%
    			\ifmygrid@coordinates%
       			\draw[/mygrid/step line]
        			\foreach \xpos in {\mygrid@xlo,...,\mygrid@xhi} {%
         			(\xpos,\mygrid@ylo) -- ++(0,-3pt)
         			node[anchor=north] {$\xpos$}
        			}
        			\foreach \ypos in {\mygrid@ylo,...,\mygrid@yhi} {%
          			(\mygrid@xlo,\ypos) -- ++(-3pt,0)
					node[anchor=east] {$\ypos$}
        			};
    			\fi%
		}
		\makeatother

%--------------------------------------
%macros
%--------------------------------------

\newcommand*{\superref}[1]{\hyperref[{#1}]{\autoref*{#1} \autopageref*{#1}}} %liens avec page et titre

\definecolor{orangelogo}{RGB}{249,125,0} %couleur orange du logo
\definecolor{bleulogo}{RGB}{11,92,180} %couleur bleue du logo

\newcommand{\circref}[1]{\CircledText{\small\textbf{\ref{#1}}} : } %référence des pastilles

	%--------------------------------------
	%redéfinition des noms
	%--------------------------------------

	\addto\captionsfrench{\renewcommand{\listfigurename}{Liste des figures}} %remplacement des titres automatiques
	\addto\captionsfrench{\renewcommand{\appendixtocname}{Annexes}}
	\addto\captionsfrench{\renewcommand{\appendixpagename}{Annexes}}
 
 	\addto\extrasfrench{%traduction des références automatiques
	\def\sectionautorefname{section}%
    \def\subsectionautorefname{sous-section}%
    \def\figureautorefname{figure}%
    \def\tableautorefname{tableau}%
    \def\exempleautorefname{exemple}%
    \def\exerciceautorefname{exercice}%
    \def\appendixautorefname{annexe}%
	}
  
	\def\frenchfigurename{{\scshape Fig.}} %style des légendes
	\def\frenchtablename{{\scshape Tab.}}
	
	%--------------------------------------
	%paramètres des tableaux
	%--------------------------------------

	\renewcommand\theadfont{\bfseries\small} %style de caractère des en-têtes des tableaux
	\renewcommand\theadalign{cc} %position des en-têtes des tableaux
	\renewcommand\theadgape{} %espacement des en-têtes des tableaux

	\newcommand{\middashrule}{\hdashline\addlinespace} %ligne pointillée de milieu de tableau
	\newcommand{\HRule}{\rule{\linewidth}{0.5mm}} %ligne fine

	%--------------------------------------
	%paramétrage mathématique
	%--------------------------------------

	\newlength{\conditionwd} %environnement de description "muibene" pour les équations et les formules
	\newenvironment{variables}[1][avec\quad] 
  		{#1\tabularx{\textwidth-\widthof{#1}}[t]{
			>{$}l<{$} @{${}:\enspace{}$}l >{ en }l >{(~}l<{~)} l@{}}}	%preambule du tableau à 5 colonnes
		{\endtabularx\\[\belowdisplayskip]}
	
	\newcommand{\pc}[1]{\SI{#1}{\percent}} %nouvelle commande pour afficher les pourcents dans SIunitx

		%--------------------------------------
		%Liste des équations
		%--------------------------------------

		\DeclareNewFloatType{equa}{placement=tbh,fileext=loe} %fileext=extension
		\floatname{equa}{\textsc{\'Eq}} 

		\newcommand{\listofequations}{%
		\listof{equa}{Liste des équations}\addcontentsline{toc}{chapter}{Liste des équations}%
		}

	%--------------------------------------
	%macros pour les listes
	%--------------------------------------

	\newlist{tabdescription}{description}{2} %liste utilisée pour les descriptions dans les tableaux (alignement sur les cellules contenant du texte)
	\setlist[tabdescription, 1]{leftmargin=*, %
	topsep=0ex, %
	parsep=0pt, %
	after=\vspace{-\baselineskip}, %
	before=\vspace{-0.75\baselineskip}}
	
	\setlist[tabdescription, 2]{nosep, %
	leftmargin=*}

	\newlist{tabitemize}{itemize}{1} %liste utilisée dans les tableaux (alignement sur les cellules contenant du texte)
	\setlist[tabitemize]{label=$-$, %
	nosep, %
	leftmargin=*, %
	 topsep=0ex, %
	 parsep=0pt, %
	 after=\vspace{-\baselineskip}, %
	 before=\vspace{-0.75\baselineskip}}

	\newlist{compactitemize}{itemize}{1} %liste compacte et sans marge
	\setlist[compactitemize]{%
	label=$-$, %
	nosep, %
	leftmargin=*}

%--------------------------------------
%Mise en page du document
%--------------------------------------

	%--------------------------------------
	%Marges
	%--------------------------------------

	\setlrmarginsandblock{25mm}{20mm}{*} %réglage marge gauche-droite
	\setulmarginsandblock{20mm}{20mm}{*} %réglage marge haut-bas
	\setheadfoot{\baselineskip}{3\baselineskip} %reglage en hauteur des en-têtes et pied-de-page
	\checkandfixthelayout

	%--------------------------------------
	%En-tête et pied-de-page
	%--------------------------------------

	\newif\ifInvnum\Invnumtrue %énorme prise de tête pour que les numéros de page de l'introduction soient disposés inversement au restant du document

	\pagestyle{plain}{} %réglage de la présence d'en-tête et de pied-de-page 
	\makeevenhead{plain}{}{}{} %en-tête page paire 
	\makeoddhead{plain}{}{}{} %en-tête page impaire 
	\makeevenfoot{plain}{%
		\ifInvnum %instruction conditionelle selon la booleenne Invnum
		\else %si Invnum faux
			{\ifFrame %instruction conditionelle selon la booleenne Frame
				\begin{tikzpicture}
					\draw node [rounded corners=3pt, draw=\BoxColor, fill=\BoxColor, text=black, inner xsep=2ex, inner ysep=5pt]{\sffamily\textbf{\thepage}};
				\end{tikzpicture}
			\else
				\sffamily\textbf{\thepage}
			\fi}
		\fi}
	{\includegraphics[scale=0.03]{logo_compagnons}}
	{%
		\ifInvnum %instruction conditionelle selon la booleenne Invnum
			\sffamily\textbf{\thepage} %si Invnumtrue
		\else
		\fi} %pied-de-page page paire

	\makeoddfoot{plain}{%
		\ifInvnum 
			\sffamily\textbf{\thepage} %si Invnumtrue
		\else
		\fi}
	{\includegraphics[scale=0.03]{logo_compagnons}}
	{%
		\ifInvnum %instruction conditionelle selon la booleenne Invnum
		\else %si Invnum faux
			{\ifFrame %instruction conditionelle selon la booleenne Frame
				\begin{tikzpicture}
					\draw node [rounded corners=3pt, draw=\BoxColor, fill=\BoxColor, text=black, inner xsep=2ex, inner ysep=5pt]{\sffamily\textbf{\thepage}};
				\end{tikzpicture}
			\else
				\sffamily\textbf{\thepage}
			\fi}
		\fi} %pied-de-page page impaire

	%--------------------------------------
	%macros générales
	%--------------------------------------

	\maxsecnumdepth{subsubsection}\setsecnumdepth{subsubsection} %numérotation des sous-sous-sections
	\setcounter{tocdepth}{3} %affichage des sous-sous-sections dans la table des matières
	\setcounter{secnumdepth}{3} %affichage de la numérotation des sous-sous-sections dans la table des matières

	\makeatletter %nouvelle commande conditionnelle interne pour différencier un chapitre en corps de texte et en annexes
		\newcommand{\inappendix}{\fi\expandafter\ifx\@chapapp\appendixname}
	\makeatother

	%--------------------------------------
	%Chapitre
	%--------------------------------------
	
		\makeatletter
	\makechapterstyle{douchy}{%
	
		\setlength{\beforechapskip}{-1cm}
		\setlength{\afterchapskip}{2cm}
		\renewcommand*{\chapnamefont}{\bfseries\sffamily\scriptsize\MakeUppercase}
		\renewcommand*{\chapnumfont}{\mdseries\sffamily}
		\renewcommand*{\chaptitlefont}{\raggedright\Huge\sffamily\bfseries}

		\renewcommand*{\printchaptername}{\raisebox{-\height}{\chapnamefont\@chapapp}}
		\renewcommand*{\chapternamenum}{\hspace{0.3cm}}
		\renewcommand*{\printchapternum}{\raisebox{-\height}{%
			\if\inappendix %instruction conditionelle pour savoir si on se situe dans les annexes ou pas (grosse prise de tête)
				\resizebox{33pt}{!}{\chapnumfont\thechapter} 
			\else
				\resizebox{27pt}{!}{\chapnumfont\thechapter}
			\fi}
		\hspace{0.2cm}}

		\settowidth{\chapindent}{%
		{\printchaptername}\chapternamenum{\printchapternum}\hspace{0.2cm}}

		\renewcommand*{\afterchapternum}{\vspace{-44.5pt}}

		\renewcommand*{\printchaptertitle}[1]{
			\if@mainmatter
				\flushright{\parbox[t]{\linewidth-\chapindent}%
				{\hrule depth 1pt\vspace{3ex}\chaptitlefont ##1}}%
				\vspace{2.75ex}%
				\hrule depth 1pt
			\else
				\parbox[t]{\textwidth}%
				{\hrule depth 1pt\vspace{3ex}\chaptitlefont ##1%
				\vspace{1.25ex}%
				\hrule depth 1pt}
			\fi}

		\renewcommand*{\afterchaptertitle}{\vskip\afterchapskip}
	}
	\makeatother
	\chapterstyle{douchy}
	
	%--------------------------------------
	%Section
	%--------------------------------------
	
	\setsecheadstyle{\huge\sffamily\bfseries\color{black}}

	%--------------------------------------
	%Sous-section
	%--------------------------------------
		
	\renewcommand{\thesubsection}{\arabic{section}. \arabic{subsection}.}
	\setsubsecheadstyle{\Large\sffamily\bfseries\color{black!75}}

	%--------------------------------------
	%Sous-sous-section
	%--------------------------------------

	\setcounter{secnumdepth}{4}
	\renewcommand{\thesubsubsection}{\arabic{section}. \arabic{subsection}. \arabic{subsubsection}.}
	\setsubsubsecheadstyle{\large\sffamily\bfseries\color{gray}} %à redéfinir une fois le cours entamé le chemin ne sera pas le même selon les cours !

%--------------------------------------
%CANEVAS
%--------------------------------------

\newcommand\BoxColor{\ifcase\thechapshift blue!30\or brown!30\or pink!30\or cyan!30\or green!30\or teal!30\or purple!30\or red!30\or olive!30\or orange!30\or lime!30\or gray!\or magenta!30\else yellow!30\fi} %définition de la couleur des marqueurs de chapitre

\newcounter{chapshift} %compteur de chapitre du marqueur de chapitre
\addtocounter{chapshift}{-1}
	
\newif\ifFrame %instruction conditionnelle pour les couleurs des pages
\Frametrue

\pagestyle{plain}

% the main command; the mandatory argument sets the color of the vertical box
\newcommand\ChapFrame{%
\AddEverypageHook{%
\ifFrame
\ifthenelse{\isodd{\value{page}}}
  {\backgroundsetup{contents={%
  \begin{tikzpicture}[overlay,remember picture]
  \node[
  	rounded corners=3pt,
    fill=\BoxColor,
    inner sep=0pt,
    rectangle,
    text width=1.5cm,
    text height=5.5cm,
    align=center,
    anchor=north west
  ] 
  at ($ (current page.north west) + (-0cm,-2*\thechapshift cm) $) %nombre négatif = espacement des marqueurs entre les différents chapitres (à régler en fin de rédaction) (4.5cm vaut un espacement équivalement à la hauteur du marqueur, une page peut en contenir 6 avec cet espacement-la mais il est le plus équilibré)
    {\rotatebox{90}{\hspace*{.5cm}%
      \parbox[c][1.2cm][t]{5cm}{%
        \raggedright\textcolor{black}{\sffamily\textbf{\leftmark}}}}};
  \end{tikzpicture}}}
  }
  {\backgroundsetup{contents={%
  \begin{tikzpicture}[overlay,remember picture]
  \node[
  	rounded corners=3pt,
    fill=\BoxColor,
    inner sep=0pt,
    rectangle,
    text width=1.5cm,
    text height=5.5cm,
    align=center,
    anchor=north east
  ] 
  at ($ (current page.north east) + (-0cm,-2*\thechapshift cm) $) %nombre négatif = espacement des marqueurs entre les différents chapitres (à régler en fin de rédaction) (4.5cm vaut un espacement équivalement à la hauteur du marqueur, une page peut en contenir 6 avec cet espacement-la mais il est le plus équilibré)
    {\rotatebox{90}{\hspace*{.5cm}%
      \parbox[c][1.2cm][t]{5cm}{%
        \raggedright\textcolor{black}{\sffamily\textbf{\leftmark}}}}};
  \end{tikzpicture}}}%
  }
  \BgMaterial%
  \fi%
}%
  \stepcounter{chapshift}
}

\renewcommand\chaptermark[1]{\markboth{\thechapter.~#1}{}} %redéfinition du marqueur de chapitre pour ne contenir que le titre du chapitre %à personnaliser selon le nombre de chapitre dans le cours

%--------------------------------------
%corps du document
%--------------------------------------

\begin{document} %corps du document
	\openleft %début de chapitre à gauche

%\end{comment}

\chapter{Unité de mesure et grandeurs physique}

\section{Généralités}

\subsection{Différences}

Cette annexe énumère les unités de mesures et de leur grandeurs physiques associées à connaître pour la maitrise des formules mathématiques en électrotechnique. Il convient de bien identifier ce qu'est une grandeur physique et une unité de mesure :
\begin{description}
\item[Unité de mesure] \'Etalon de mesure nécessaire pour la mesure d'une grandeur physique dont le fondement est l'exacte reproductibilité expérimentale de l'étalon\,;
\item[Grandeur physique] Toute propriété des sciences de la nature qui peut être mesurée ou calculées et dont les différentes valeurs s'expriment à l'aide d'une nombre réel ou complexe. Une grandeur physique peut s'exprimer sans unité de mesure, ce sont des \emph{grandeurs sans dimension}. Mais l'inverse n'est pas vraie, toute unité de mesure est associée une grandeur physique.\\
La notion générale de grandeur physique peut être divisées en des notions plus précises, indiquée au moyen d'indices ou d'un symbole usuel différent.
\item[Dimension] Expression de la dépendance d'une grandeur par rapport aux grandeurs de base d'un système de grandeurs sous la forme d'un produit de puissance de facteurs correspondant aux grandeurs de base, en omettant tout facteur numérique.
\end{description}
Les tableaux situés en \superref{subsec:systeme_international} sont issus des normes ISO 80000-xx\cite{ISO:80000-2013}, les normes internationales régissant le Système International de grandeurs (\emph{International System of Quantities}, ISQ), qui font également le lien avec le Système International d'unités (SI).

\subsection{Quelques règles de rédaction}

\begin{description}
	\item[Symboles des grandeurs] Les symboles usuels des grandeurs prennent généralement la forme d'une seule lettre (alphabet grec ou latin), toujours en italique, et peuvent être précisés par des indices.
	\item [Indice] Un indice permet de différencier des grandeurs présentant le même symbole usuel ou, pour une même grandeur, différentes applications de celle-ci.
		\begin{itemize}
			\item Symbole d'une grandeur physique ou d'une variable mathématique\,;
			\item Mots ou nombres fixes.
		\end{itemize}
	\item[Symboles des unités] Les symboles des unités prennent généralement la forme d'une seule lettre (alphabet grec ou latin), toujours en caractère droit, ce qui permet de les différencier des symboles des grandeurs.\\
Une unité composée d'une multiplication de deux unités ou plus peut être indiquée de deux manières :
		\begin{gather*}
			\newton\cdot\metre \\ 
			\newton\metre
		\end{gather*}
Il convient de faire attention lorsque le symbole d'une unité est le même que celui d'un préfixe.
\end{description}

\subsection{Terminologie}

\begin{description}
\item[Coefficient] Dans une équation type $A=k \cdot B$, $k$ est le coefficient/facteur et $A$ est une grandeur proportionnelle à $B$. Usage du terme \emph{coefficient} (ou \emph{module}) lorsque les grandeurs $A$ et $B$ présentent des \emph{dimensions} différentes.
\item[Facteur] Dans une équation type $A=k \cdot B$, $k$ est le coefficient/facteur et $A$ est une grandeur proportionnelle à $B$. Usage du terme \emph{facteur} lorsque les grandeurs $A$ et $B$ sont de même \emph{dimension}.
\item[Paramètre] Combinaison de grandeurs qui apparaissent sous une telle forme dans les équations, pouvant être considérée comme constituant de nouvelles grandeurs.
\item[Nombre] Combinaison de grandeurs sans dimension.
\item[Rapport] Quotient sans dimension de deux grandeurs.
\item[Constante] Grandeur qui présente la même valeur en toutes circonstances.
\item[Massique] Adjectif apposé à une grandeur caractérisant le quotient de cette grandeur par la masse.
\item[Volumique] Adjectif apposé à une grandeur caractérisant le quotient de cette grandeur par le volume.
\item[Surfacique] Adjectif apposé à une grandeur caractérisant le quotient de cette grandeur par l'aire.
\item[Densité] Adjectif apposé à une grandeur exprimant un flux ou un courant, qui caractérise le quotient de cette grandeur par l'aire.
\item[Linéique] Adjectif apposé à une grandeur caractérisant le quotient de cette grandeur par la longueur.
\item[Molaire] Adjectif apposé à une grandeur caractérisant le quotient de cette grandeur par la quantité de matière.
\item[Concentration] Adjectif apposé à une grandeur, spécifiquement dans le cas d'un mélange, caractérisant le quotient de cette grandeur par le volume total.
\end{description}

\subsection{Alphabet}

\begin{table}[h!]
\caption{Alphabet grec}
\begin{tabularx}{\textwidth}[t]{l X X X X l X X X X}
\cmidrule[\heavyrulewidth](lr){1-5} \cmidrule[\heavyrulewidth](lr){6-10}
\thead{Nom} 		& \multicolumn{2}{c}{\thead{Caractère\\romain}} 	& \multicolumn{2}{c}{\thead{Caractère\\italique}} & \thead{Nom} 		& \multicolumn{2}{c}{\thead{Caractère\\romain}} 	& \multicolumn{2}{c}{\thead{Caractère\\italique}} \\
\cmidrule[\lightrulewidth](lr){1-5} \cmidrule[\lightrulewidth](lr){6-10}
alpha 					& A						& $\alphaup$											& \textit{A}							& $\alpha$							& nu							& N						& $\nuup$												& \textit{N}							& $\nu$ \\
beta 						& B						& $\betaup$											& \textit{B}							& $\beta$  							& xi							& $\Xiup$				& $\xiup$												& $\mathit{\Xi}$					& $\xi$ \\
gamma 				& $\Gammaup$		& $\gammaup$										& $\mathit{\Gamma}$			& $\gamma$ 							& omicron					& O						& o														& \textit{O}							& \textit{o} \\
delta						& $\Deltaup$			& $\deltaup$											& $\mathit{\Delta}$				& $\delta$ 							& pi							& $\Piup$				& $\piup$, $\varpiup$								& $\mathit{\Pi}$					& $\pi$, $\varpi$ \\
epsilon					& E						& $\epsilonup$, $\varepsilonup$				& \textit{E}							& $\epsilon$, $\varepsilon$ 	& rhô						& P						& $\rhoup$, $\varrhoup$						& \textit{P}							& $\rho$, $\varrho$ \\
zêta						& Z						& $\zetaup$											& \textit{Z}							& $\zeta$ 								& sigma					& $\Sigmaup$		& $\sigmaup$										& $\mathit{\Sigma}$				& $\sigma$ \\
êta						& H						& $\etaup$											& \textit{H}							& $\eta$								& tau						& T						& $\tauup$											& \textit{T}							& $\tau$ \\
thêta						& $\Thetaup$		& $\thetaup$, $\varthetaup$					& $\mathit{\Theta}$				& $\theta$, $\vartheta$	 		& upsilon					& Y						& $\upsilonup$										& \textit{Y}							& $\upsilon$ \\
iota						& I						& $\iotaup$											& \textit{I}							& $\iota$ 								& phi						& $\Phiup$			& $\phiup$											& $\mathit{\Phi}$					& $\phi$ \\
kappa					& K						& $\kappaup$, $\varkappaup$				& \textit{K}							& $\kappa$, $\varkappa$ 		& khi						& X						& $\chiup$												& \textit{X}							& $\chi$ \\
lambda					& $\Lambdaup$		& $\lambdaup$										& $\mathit{\Lambda}$			& $\lambda$							& psi						& $\Psiup$				& $\psiup$												& $\mathit{\Psi}$					& $\psi$ \\
mu						& M						& $\muup$											& \textit{M}							& $\mu$								& oméga					& $\Omegaup$		& $\omegaup$										& $\mathit{\Omega}$			& $\Omega$ \\
\cmidrule[\heavyrulewidth](lr){1-5} \cmidrule[\heavyrulewidth](lr){6-10}
\end{tabularx}
\end{table}


\subsection{Système International}
\label{subsec:systeme_international}

\subsubsection{Généralités}

Le Système International d'unités est un système cohérent d'unités dans l'\emph{ISQ}. Il est toujours \emph{SI} dans toutes les langues et est formé de :
\begin{itemize}
\item Sept unités de base\,;
\item Des unités dérivées de ces unités de base.
\end{itemize}

\subsubsection{Unités SI et grandeurs}

\begin{table}[!h]
\caption{Unités SI et grandeurs correspondante de base\label{tab:unites_SI_base}}
\begin{tabularx}{\textwidth}{X R X R}
\toprule
\multicolumn{2}{c}{\thead{Grandeur de base de l'ISQ}} & \multicolumn{2}{c}{\thead{Unité SI de base}} \\
\cmidrule(lr){1-2} \cmidrule(lr){3-4} 
\thead[l]{Nom} & \thead[r]{Symbole usuel} & \thead[l]{Nom} & \thead[r]{Symbole} \\
\midrule 
Longueur 									& $L$ 			& mètre 			& \meter \\
Masse										& $M, m$ 		& kilogramme 	& \kilogram \\
Temps										& $T$			& seconde			& \second \\
Courant électrique 					& $I$				& ampère			& \ampere \\
Température thermodynamique	& $\Theta$	& kelvin				& \kelvin \\
Quantité de matière					& $N$			& mole				& \mole \\
Intensité lumineuse					& $J$				& candela			& \candela \\
\bottomrule
\end{tabularx}
\end{table}

\begin{table}[!h]
\caption{Préfixes des unités SI \label{tab:prefixes_unites_SI}}
\begin{tabularx}{\textwidth}{C X R C X R}
\cmidrule[\heavyrulewidth](lr){1-3} \cmidrule[\heavyrulewidth](lr){4-6} 

\multirow[c]{2}{*}{\thead{Facteur}} 	& \multicolumn{2}{c}{\thead{Préfixe}} 	& \multirow[c]{2}{*}{\thead{Facteur}} 	& \multicolumn{2}{c}{\thead{Préfixe}} \\

\cmidrule(lr){2-3} \cmidrule(lr){5-6} 

															& \thead{Nom} 		& \thead{Symbole}	& 																& \thead{Nom} 		& \thead{Symbole} \\
\cmidrule[\lightrulewidth](lr){1-3} \cmidrule[\lightrulewidth](lr){4-6} 
$10^{24}$											& yotta 						& Y		& $10^{-1}$											& déci 						& d  \\
$10^{21}$											& zetta 						& Z  		& $10^{-2}$											& centi 						& c  \\ 
$10^{18}$											& exa 						& E  		&															&								&	\\ 
$10^{15}$											& péta 						& P  		& $10^{-3}$											& milli 						& m  \\ 
															&								&			& $10^{-6}$											& micro 					& $\mu$  \\ 
$10^{12}$											& téra 						& T  		& $10^{-9}$											& nano 						& n  \\
$10^{9}$												& giga 						& G  		& $10^{-12}$										& pico 						& p  \\
$10^{6}$												& méga 					& M  		& 															&								& \\
$10^{3}$												& kilo 						& k 		& $10^{-15}$										& femto 					& f  \\
															&								&			& $10^{-18}$										& atto 						& a  \\
$10^{2}$												& hecto 					& h		& $10^{-21}$										& zepto 					& z \\
$10^{1}$												& déca	 					& da 		& $10^{-24}$										& yocto	 					& y \\
\cmidrule[\heavyrulewidth](lr){1-3} \cmidrule[\heavyrulewidth](lr){4-6} 
\end{tabularx}
\end{table}

\begin{xltabular}{\textwidth}{X r X M}
\caption{Unités SI dérivées avec des noms et des symboles spéciaux\label{tab:unites_SI_derivee}}\\
\toprule
\multicolumn{2}{c}{\thead{Grandeur dérivée de l'ISQ}} & \multicolumn{3}{c}{\thead{Unité SI dérivée}} \\
\cmidrule(lr){1-2} \cmidrule(lr){3-5}
\thead[l]{Nom} & \thead[r]{Symbole usuel} & \thead[l]{Nom} & \multicolumn{2}{c}{\thead[c]{Symbole \& Valeur}} \\
\midrule %filet de milieu de tableau
\endfirsthead %en-tête de la première page du tableau  
\multicolumn{5}{l}{\small\textit{Page précédente}} \\
\midrule %filet de milieu de tableau
\multicolumn{2}{c}{\thead{Grandeur dérivée de l'ISQ}} & \multicolumn{3}{c}{\thead{Unité SI dérivée}} \\
\cmidrule(lr){1-2} \cmidrule(lr){3-5}
\thead[l]{Nom} & \thead[r]{Symbole usuel} & \thead[l]{Nom} & \multicolumn{2}{l}{\thead[l]{Symbole \& Valeur}} \\
\midrule %filet de milieu de tableau
\endhead %en-tête de la première page du tableau  
\midrule %filet de milieu de tableau
\multicolumn{5}{r}{\small\textit{Page suivante}} \\
\endfoot %pied de page de toutes les pages du tableau
\bottomrule
\endlastfoot %pied de page de la dernièredu tableau
Angle plan											& $\alpha$ 						& radian 			& \radian 					& \si{\meter\per\meter} \\
Angle solide										& $\Omega$						& stéradian	 	&	\steradian		 		& 	\si{\square\meter\per\square\meter} \\
Fréquence 										& $f$									& hertz				&	\hertz 					&	\si{\per\second} \\
Force												& $F$								& newton			&	\newton					&  \si{\kilogram\meter\per\square\second} \\
Pression, contrainte							& $P$								& pascal			&	\pascal					& 	\si{\newton\per\square\meter} \\
\'Energie, travail								& $W$								& joule				& 	\joule					& 	\si{\kilo\gram\square\meter\per\square\second} \\
Puissance											& $P$								& watt				& 	\watt						&	\si{\joule\per\second} \\
Charge électrique								& $Q$								& coulomb			& 	\coulomb				&	\si{\ampere\second} \\
Différence de potentiel électrique		& $U, V$							& volt				& 	\volt						&	\si{\watt\per\ampere} \\
Capacité électrique							& $C$								& farad				& 	\farad					& \si{\coulomb\per\volt} \\
Résistance électrique							& $R$								& ohm				& 	\ohm						& \si{\volt\per\ampere} \\
Conductance électrique						& $G$								& siemens			&	\siemens				& \si{\per\ohm} \\
Flux magnétique								& $\Phi$							& weber			&	\weber					& \si{\volt\second} \\
Induction magnétique						& $\overrightarrow{B}$		& tesla				& \tesla						& \si{\weber\per\square\meter} \\
Inductance										& $L$								& henry				& \henry					& \si{\weber\per\ampere} \\
Température Celsius							& $T$								& celsius			& \celsius					& \kelvin - 273,15 \\
Flux lumineux									& $J$									& lumen			& \lumen					& \si{\candela\steradian} \\
\'Eclairement lumineux						& $E, E_v$							& lux					& \lux						& \si{\lumen\per\square\meter} \\
\end{xltabular}

\begin{table}[!h]
\caption{Unités en usage avec le SI\label{tab:unites_usage_SI}}
\begin{tabularx}{\textwidth}{X r X M}
\toprule
\multicolumn{2}{c}{\thead{Grandeur}} & \multicolumn{3}{c}{\thead{Unités}} \\
\cmidrule(lr){1-2} \cmidrule(lr){3-5} 
\thead[l]{Nom} & \thead[r]{Symbole usuel} & \thead[l]{Nom} & \multicolumn{2}{l}{\thead[l]{Symbole \& Valeur}} \\
\midrule
Temps											& $t$								 	& minute 			& \minute 				& \SI{60}{\second} \\
													& 											& heure				& \hour					& \SI{60}{\minute} \\
													&											&	jour				& \si{\day}			& \SI{24}{\hour} \\
\addlinespace
Angle plan										& $\alpha$							& degré				& \degree				& \sfrac{180}{\pi}\times\radian \\
													&											& minute			& \arcminute			& \sfrac{1}{60}\times\degree \\
													&											& seconde			& \arcsecond			& \sfrac{1}{60}\times\arcminute \\
\addlinespace
Volume											& $V$									& litre				& \litre, \liter			& \si{\cubic\deci\metre} \\
\addlinespace
Masse											& $M, m$								& tonne				& \tonne				& \SI{1000}{\kilo\gram} \\
\bottomrule
\end{tabularx}
\end{table}

\begin{table}[H]
\caption{Unités en usage avec le SI dont la valeur est obtenue expérimentalement\label{tab:unites_SI_experimentales}}
\begin{tabularx}{\textwidth}{X r X M}
\toprule
\multicolumn{2}{c}{\thead{Grandeur}} 		& \multicolumn{3}{c}{\thead{Unités}} \\
\cmidrule(lr){1-2} \cmidrule(lr){3-5} 
\thead[l]{Nom} & \thead[r]{Symbole\\usuel} 	& \thead[l]{Nom} 	& \multicolumn{2}{c}{\thead[c]{Symbole \& Valeur}} \\
\midrule
\'Energie 		& $W$ 			& électronvolt 					& \multicolumn{2}{p{8cm}}{\'Energie cinétique acquise par un électron en traversant une différence de potentiel de 1v dans le vide.} \\%
					& 					& 										& \electronvolt					& \SI{1,602176634e-19}{\joule} \\
\addlinespace
Masse			& $M, m$		& dalton							& \multicolumn{2}{p{8cm}}{$\sfrac{1}{12}$ de la masse d'un atome du nucléide \ce{^{12}C} au repos et à l'état fondamental.} \\%
					& 					& 										& \dalton							& \SI{1,660538782e-27}{\kilo\gram} \\
\addlinespace
Longueur		& $L$			& unité astronomique			& \multicolumn{2}{p{8cm}}{Valeur conventionnelle approximativement égale à la valeur moyenne de la distance entre le Soleil et la Terre.} \\%
					& 					& 										& \astronomicalunit			& \SI{1,49597870691e11}{\metre} \\
\bottomrule
\end{tabularx}
\end{table}

\section{Mathématique}

Le tableau suivant est extrait de l'ouvrage \citetitle{BourgeoisCogniel2005}, il référence les notations mathématiques utilisées en électrotechnique.

\begin{table}[!h]
\caption{Signe et symbole mathématique\label{tab:signes_mathematiques}}
\renewcommand{\tabularxcolumn}[1]{m{#1}} %centre verticalement toutes les cellules
\begin{tabularx}{\textwidth}{j j X j j X}
\cmidrule[\heavyrulewidth](lr){1-3} \cmidrule[\heavyrulewidth](lr){4-6}
\multicolumn{1}{c}{\thead{Signe}} 		& \multicolumn{1}{c}{\thead{Utilisation}}		& \multicolumn{1}{c}{\thead{\'Enoncé}} 		& \multicolumn{1}{c}{\thead{Signe}} 		& \multicolumn{1}{c}{\thead{Utilisation}}		& \multicolumn{1}{c}{\thead{\'Enoncé}} \\
\cmidrule[\lightrulewidth](lr){1-3} \cmidrule[\lightrulewidth](lr){4-6} 
=					& a=b				& $a$ égal $b$											& +					& a+b														& $a$ plus $b$ \\
\neq				& a\neq b			& $a$ est différent de $b$ 							& -					& a-b															& $a$ moins $b$ \\
\triangleq		& a\triangleq b	& $a$ correspond à $b$								& \times			& a\times b												& $a$ multiplié par $b$ \\
\simeq			& a\simeq b		& $a$ est approximativement égal à $b$ 	& \cdot				& a\cdot  b												& \\
<					& a < b				& $a$ est strictement inférieur à $b$ 			&						& a\ b														& \\
>					& a > b				& $a$ est strictement supérieur à $b$ 		& \frac\				& \frac{a}{b}											& $a$ divisé par $b$ \\
\leq				& a\leq b			& $a$ est inférieur ou égal à $b$				& /					& a/b															& \\
\geq				& a\geq b			& $a$ est supérieur ou égal à $b$				& \sum				& {\displaystyle\sum_{i=1}^{n} a_{i}} 	& $a_{1} + a_{2} + a_{3} \ldots a_{n}$ \\
\ll					& a\ll b				& $a$ est très inférieur ou égal à $b$			& \prod				& {\displaystyle\prod_{i=1}^{n} a_{i}}	& $a_{1} \times a_{2} \times a_{3} \ldots a_{n}$ \\
\gg				& a\gg b			& $a$ est très supérieur ou égal à $b$		& !					& n!															& $1 \times 2 \times 3 \ldots n$ \\
\infty				&						& infini														& 						& a^{n}													& $a$ puissance $n$ \\
\pm				& a\pm b			& $a$ plus ou moins $b$ 							& 						& \sqrt{a}													& racine carrée de $a$ \\
\in					& x\in A				& $x$ appartient à $a$								& 						& \sqrt[n]{a}												& racine n\ieme de $a$ \\
\notin			& x\notin A 		& $x$ n'appartient pas à $a$						& 						& a^{1/n}													& \\
\vert\ \vert\	& \vert a \vert	& valeur absolue de $a$ 							&						& 																& \\
\cmidrule[\heavyrulewidth](lr){1-3} \cmidrule[\heavyrulewidth](lr){4-6}
\end{tabularx}
\end{table}

\begin{xltabular}{\textwidth}{i X i X}
\caption{Signe et symbole mathématique\label{tab:symbole_mathematiques}} \\
\cmidrule[\heavyrulewidth](lr){1-2} \cmidrule[\heavyrulewidth](lr){3-4} 
\multicolumn{1}{c}{\thead{Symbole}} 		& \multicolumn{1}{c}{\thead{\'Enoncé}}		& \multicolumn{1}{c}{\thead{Symbole}} 		& \multicolumn{1}{c}{\thead{\'Enoncé}} \\
\cmidrule[\lightrulewidth](lr){1-2} \cmidrule[\lightrulewidth](lr){3-4}
\endfirsthead
\multicolumn{4}{l}{\small\textit{Page précédente}} \\
\cmidrule[\lightrulewidth](lr){1-2} \cmidrule[\lightrulewidth](lr){3-4}
\multicolumn{1}{c}{\thead{Symbole}} 		& \multicolumn{1}{c}{\thead{\'Enoncé}}		& \multicolumn{1}{c}{\thead{Symbole}} 		& \multicolumn{1}{c}{\thead{\'Enoncé}} \\
\cmidrule[\lightrulewidth](lr){1-2} \cmidrule[\lightrulewidth](lr){3-4}
\endhead
\cmidrule[\lightrulewidth](lr){1-2} \cmidrule[\lightrulewidth](lr){3-4}
\multicolumn{4}{r}{\small\textit{Page suivante}} \\
\endfoot
\cmidrule[\heavyrulewidth](lr){1-2} \cmidrule[\heavyrulewidth](lr){3-4} 
\endlastfoot
f													& fonction ou application 																& \cos x						&	cosinus de $x$ \\
f(x)												& valeur de la fonction $f$ respectivement en $x$ 							& \sin x						&	sinus de $x$ \\
\left[ f(x) \right] ^{a} _{b}				& $f(b)-f(a)$																					& \tan x						&	tangente de $x$ \\
f(x)\ | ^{a} _{b} 							& 																									& \cot x						&	cotangente de $x$ \\
\lim\limits_{x \rightarrow a} f(x)		& limite de $f(x)$ quand $x$ tend vers $a$ 									& \arccos x				& réciproque du cosinus de $x$ \\
f\prime											& dérivée (première) de la fonction $f$											& \arcsin x					& inverse du sinus de $x$ \\
f^{(k)}(x)										& dérivée d'ordre $k$ de la fonction $f$ 											& \arctan x				& inverse de la tangente de $x$ \\
\Delta f											& dérivée totale globale de la fonction $f$\supercite{Wiki:NDS} 		& e = 2,7182818		& base des logarithmes népériens \\
\frac{df}{dx}								& dérivée totale locale de la fonction $f$ par rapport à $x$ 				& \exp x					& exponentielle de base $e$ de $x$ \\
\frac{\partial f}{\partial x}				& dérivée partielle locale de la fonction $f$ par rapport à $x$			& \ln x						& logarithme népérien de $x$ \\
\frac{\delta f}{\delta x}					& variation élémentaire de la fonction $f$ par rapport à $x$				& \lg x 						& logarithme décimal de $x$ \\
\int_{a}^b f(x)dx							& intégrale définie de la fonction $f$ de $a$ à $b$ 							& \mathrm{i\ ou\ j}	& unité imaginaire \\
\int_{a}^b f(x)dx							& intégrale définie de la fonction $f$ de $a$ à $b$ 							& \arg 						& argument \\					
\mathbb{N}									& ensemble des entiers naturels 													& \mathbb{R}			& ensemble des entiers réels \\
\mathbb{Z}									& ensemble des entiers 																	& \mathbb{C}			& ensemble des nombres complexes \\
\mathbb{Q}									& ensemble des nombres rationnels												& \mathbb{P}			& ensemble des nombres premiers \\
\end{xltabular}


\section{Espace \& temps}

\subsection{Généralités}

Les tableaux suivants détaillent les noms, les symboles et les définitions des unités et grandeurs utilisées pour décrire mathématiquement l'espace et le temps. Ces notations seront d'applications pour le restant des cours.

\begin{xltabular}{\textwidth}{l k l k X}
\caption{Unités SI et grandeurs définissant l'espace et le temps\label{tab:unites_espace_temps}} \\
\toprule
\multicolumn{2}{c}{\thead{Grandeur}} & \multicolumn{2}{c}{\thead{Unité}} & \multirow[c]{2}{*}{\thead[c]{Remarque}} \\
\cmidrule(lr){1-2} \cmidrule(lr){3-4}
\thead[l]{Nom} & \multicolumn{1}{r}{\thead[r]{Symbole usuel}} & \thead[l]{Nom} & \multicolumn{1}{r}{\thead[r]{Symbole}} & \\
\midrule %filet de milieu de tableau
\endfirsthead %en-tête de la première page du tableau  
\multicolumn{5}{l}{\small\textit{Page précédente}} \\
\midrule %filet de milieu de tableau
\multicolumn{2}{c}{\thead{Grandeurs}} & \multicolumn{2}{c}{\thead{Unités}} & \multirow[c]{2}{*}{\thead[c]{Remarques}} \\
\cmidrule(lr){1-2} \cmidrule(lr){3-4}
\thead[l]{Nom} & \multicolumn{1}{r}{\thead[r]{Symbole usuel}} & \thead[l]{Nom} & \multicolumn{1}{r}{\thead[r]{Symbole}} & \\
\midrule %filet de milieu de tableau
\endhead %en-tête de la première page du tableau  
\midrule %filet de milieu de tableau
\multicolumn{5}{r}{\small\textit{Page suivante}} \\
\endfoot %pied de page de toutes les pages du tableau
\bottomrule
\endlastfoot %pied de page de la dernièredu tableau
Longueur 			& L, l				& Mètre				& \metre				& \\
\end{xltabular}

\end{document}
