%--------------------------------------
%ELECTROTECHNIQUE - SCHEMA DE LIAISON A LA TERRE
%--------------------------------------

%utiliser les environnement \begin{comment} \end{comment} pour mettre en commentaire le préambule une fois la programmation appelée dans le document maître (!ne pas oublier de mettre en commentaire \end{document}!)

\begin{comment}

\documentclass[a4paper, 11pt, twoside, fleqn]{memoir}

\usepackage{AOCDTF}

%--------------------------------------
%CANEVAS
%--------------------------------------

\newcommand\BoxColor{\ifcase\thechapshift blue!30\or brown!30\or pink!30\or cyan!30\or green!30\or teal!30\or purple!30\or red!30\or olive!30\or orange!30\or lime!30\or gray!\or magenta!30\else yellow!30\fi} %définition de la couleur des marqueurs de chapitre

\newcounter{chapshift} %compteur de chapitre du marqueur de chapitre
\addtocounter{chapshift}{-1}
	
\newif\ifFrame %instruction conditionnelle pour les couleurs des pages
\Frametrue

\pagestyle{plain}

% the main command; the mandatory argument sets the color of the vertical box
\newcommand\ChapFrame{%
\AddEverypageHook{%
\ifFrame
\ifthenelse{\isodd{\value{page}}}
  {\backgroundsetup{contents={%
  \begin{tikzpicture}[overlay,remember picture]
  \node[
  	rounded corners=3pt,
    fill=\BoxColor,
    inner sep=0pt,
    rectangle,
    text width=1.5cm,
    text height=5.5cm,
    align=center,
    anchor=north west
  ] 
  at ($ (current page.north west) + (-0cm,-2*\thechapshift cm) $) %nombre négatif = espacement des marqueurs entre les différents chapitres (à régler en fin de rédaction) (4.5cm vaut un espacement équivalement à la hauteur du marqueur, une page peut en contenir 6 avec cet espacement-la mais il est le plus équilibré)
    {\rotatebox{90}{\hspace*{.5cm}%
      \parbox[c][1.2cm][t]{5cm}{%
        \raggedright\textcolor{black}{\sffamily\textbf{\leftmark}}}}};
  \end{tikzpicture}}}
  }
  {\backgroundsetup{contents={%
  \begin{tikzpicture}[overlay,remember picture]
  \node[
  	rounded corners=3pt,
    fill=\BoxColor,
    inner sep=0pt,
    rectangle,
    text width=1.5cm,
    text height=5.5cm,
    align=center,
    anchor=north east
  ] 
  at ($ (current page.north east) + (-0cm,-2*\thechapshift cm) $) %nombre négatif = espacement des marqueurs entre les différents chapitres (à régler en fin de rédaction) (4.5cm vaut un espacement équivalement à la hauteur du marqueur, une page peut en contenir 6 avec cet espacement-la mais il est le plus équilibré)
    {\rotatebox{90}{\hspace*{.5cm}%
      \parbox[c][1.2cm][t]{5cm}{%
        \raggedright\textcolor{black}{\sffamily\textbf{\leftmark}}}}};
  \end{tikzpicture}}}%
  }
  \BgMaterial%
  \fi%
}%
  \stepcounter{chapshift}
}

\renewcommand\chaptermark[1]{\markboth{\thechapter.~#1}{}} %redéfinition du marqueur de chapitre pour ne contenir que le titre du chapitre %à personnaliser selon le nombre de chapitre dans le cours

%--------------------------------------
%corps du document
%--------------------------------------

\begin{document} %corps du document
	\openleft %début de chapitre à gauche

\end{comment}

\begin{figure}[h]
\caption{Boucle de défaut du courant $I_d$ sur L1}
\begin{circuitikz}[circuit ee IEC relay]
%\DrawGrid{(-1,-5)}{(9,3)} %grille d'aide pour le placement des objets

%alimentation

\node (D1) [make contact=point left, circuit breaker={point left}, tiny circuit symbols, activated] at (1,0.45) {};
\node (T1) [oosourcetransshape, prim=delta,sec=wye] at (0,0) {};


%neutre/terre

\node (RN) [R, label=$R_B$, rotate=90, tiny circuit symbols] at (0,-2.7) {};
\node (G1) [tlground] at (0,-3.9) {};
\draw [green!, thick] (G1) to node {} (RN) ; 
\draw [green!, thick] (RN) to (0,-0.5) to node {} (T1.sec4) ; 
\draw [dashed, yellow!, thick] (G1) to node {} (RN) ;
\draw [dashed, yellow!, thick] (RN) to (0,-0.5) to node {} (T1.sec4) ;

\node (RT) [resistor, rotate=90, tiny circuit symbols, label=$R_A$] at (2.5,-2.7) {};
\draw[-triangle 45, red] (2.8,-2) -- (2.8,-1) node[right,midway] {$U_d$};
\node (G2) [tlground] at (2.5,-3.9) {};
\draw [green!, thick] (RT) to (G2); 
\draw [dashed, yellow!, thick] (RT) to (G2);
\node (G2) [tlground] at (2.5,-3.9) {};
\draw [green!, thick] (G1) to (0,-4.2) to (2.5,-4.2) to (G2);
\draw [dashed, yellow!, thick] (G1) -- (0,-4.2) -- (2.5,-4.2) node [midway,below] {\color{black}$I_d$} -- (G2);
\node (G1) [tlground] at (0,-3.9) {};
\node (G2) [tlground] at (2.5,-3.9) {};

%appareil 1

\node (C2) [circ, scale=0.5] at (2.5,0.45) {};
\node (RD) [resistor, label=$R_d$, rotate=90, tiny circuit symbols] at (2.5,-1.5) {};

\draw [green!, thick] (RD) to (RT); 
\draw [dashed, yellow!, thick] (RD) to (RT); 

\draw [brown, thick] (T1.sec1) to (0.5,0.45) to (D1) to (C2) to (RD);
\node (T1) [oosourcetransshape, prim=delta,sec=wye] at (0,0) {};

%chemin courant

\fill [yellow!, decoration=lightning bolt, decorate] (2.5,-1.2) -- ++ (0.5,0.8); %éclairs
\path [postaction={on each segment={mid arrow=red}}]  (T1.sec1) -- (0.5,0.45) -- (D1) -- (C2) -- (RD) -- (RT) -- (G2) -- (2.5,-4.2) -- (1.666,-4.2) -- (0.88888,-4.2)  -- (0,-4.2) -- (G1) -- (RN) -- (0,-0.5) -- (T1.sec4); 

\callout{1,-0.5}{\cstep\label{pas:1}}{2.4,-1.2};


\end{circuitikz}
\end{figure}



%\end{document}

