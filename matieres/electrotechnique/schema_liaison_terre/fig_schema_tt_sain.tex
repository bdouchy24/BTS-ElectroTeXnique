%--------------------------------------
%ELECTROTECHNIQUE - SCHEMA DE LIAISON A LA TERRE
%--------------------------------------

%utiliser les environnement \begin{comment} \end{comment} pour mettre en commentaire le préambule une fois la programmation appelée dans le document maître (!ne pas oublier de mettre en commentaire \end{document}!)

\begin{comment}

\documentclass[a4paper, 11pt, twoside, fleqn]{memoir}

\usepackage{AOCDTF}

%--------------------------------------
%CANEVAS
%--------------------------------------

\newcommand\BoxColor{\ifcase\thechapshift blue!30\or brown!30\or pink!30\or cyan!30\or green!30\or teal!30\or purple!30\or red!30\or olive!30\or orange!30\or lime!30\or gray!\or magenta!30\else yellow!30\fi} %définition de la couleur des marqueurs de chapitre

\newcounter{chapshift} %compteur de chapitre du marqueur de chapitre
\addtocounter{chapshift}{-1}
	
\newif\ifFrame %instruction conditionnelle pour les couleurs des pages
\Frametrue

\pagestyle{plain}

% the main command; the mandatory argument sets the color of the vertical box
\newcommand\ChapFrame{%
\AddEverypageHook{%
\ifFrame
\ifthenelse{\isodd{\value{page}}}
  {\backgroundsetup{contents={%
  \begin{tikzpicture}[overlay,remember picture]
  \node[
  	rounded corners=3pt,
    fill=\BoxColor,
    inner sep=0pt,
    rectangle,
    text width=1.3cm,
    text height=5.5cm,
    align=center,
    anchor=north west
  ] 
  at ($ (current page.north west) + (-0cm,-2*\thechapshift cm) $) %nombre négatif = espacement des marqueurs entre les différents chapitres (à régler en fin de rédaction) (4.5cm vaut un espacement équivalement à la hauteur du marqueur, une page peut en contenir 6 avec cet espacement-la mais il est le plus équilibré)
    {\rotatebox{90}{\hspace*{.3cm}%
      \parbox[c][1.2cm][t]{5cm}{%
        \raggedright\textcolor{black}{\sffamily\textbf{\leftmark}}}}};
  \end{tikzpicture}}}
  }
  {\backgroundsetup{contents={%
  \begin{tikzpicture}[overlay,remember picture]
  \node[
  	rounded corners=3pt,
    fill=\BoxColor,
    inner sep=0pt,
    rectangle,
    text width=1.3cm,
    text height=5.5cm,
    align=center,
    anchor=north east
  ] 
  at ($ (current page.north east) + (-0cm,-2*\thechapshift cm) $) %nombre négatif = espacement des marqueurs entre les différents chapitres (à régler en fin de rédaction) (4.5cm vaut un espacement équivalement à la hauteur du marqueur, une page peut en contenir 6 avec cet espacement-la mais il est le plus équilibré)
    {\rotatebox{90}{\hspace*{.3cm}%
      \parbox[c][1.2cm][t]{5cm}{%
        \raggedright\textcolor{black}{\sffamily\textbf{\leftmark}}}}};
  \end{tikzpicture}}}%
  }
  \BgMaterial%
  \fi%
}%
  \stepcounter{chapshift}
}

\renewcommand\chaptermark[1]{\markboth{\thechapter.~#1}{}} %redéfinition du marqueur de chapitre pour ne contenir que le titre du chapitre %à personnaliser selon le nombre de chapitre dans le cours

%--------------------------------------
%corps du document
%--------------------------------------

\begin{document} %corps du document
	\openleft %début de chapitre à gauche

\end{comment}

\begin{circuitikz}[circuit ee IEC relay]
%\DrawGrid{(-1,-5)}{(9,3)} %grille d'aide pour le placement des objets

%sol

\fill [gray!50] (-1,-3.5) -- (5.5,-3.5) -- (5.5,-3.7) -- (-1,-3.7) -- cycle;
\draw [thick] (-1,-3.5) -- (5.5,-3.5);

%alimentation

\node (T1) [oosourcetransshape,prim=delta,sec=wye] at (0,0) {};
\node (D1) [make contact=point left, circuit breaker={point left}, tiny circuit symbols] at (1,0.45) {};
\node (D2) [make contact=point left, circuit breaker={point left}, tiny circuit symbols] at (1,0.15) {};
\node (D3) [make contact=point left, circuit breaker={point left}, tiny circuit symbols] at (1,-0.15) {};
\node (D4) [make contact=point left, circuit breaker={point left}, tiny circuit symbols] at (1,-0.45) {};

\draw [rounded corners=0.2cm] (1.4, 0.6) rectangle (1.8,-0.6);
\draw [dashed, rounded corners, thin]  (D1) to (1,-0.8) to (1.6,-0.8) to (1.6,-0.6);

\draw [brown] (-1,0.3) to (-0.5,0.3) to node {} (T1.prim1);
\draw [black] (-1,0) to (-0.5,0) to node {} (T1.prim2);
\draw [gray] (-1,-0.3) to (-0.5,-0.3) to node {} (T1.prim3);

\draw [brown] (5.5,0.45) to (D1) to (0.5,0.45) to (T1.sec1);
\draw [black] (5.5,0.15) to (D2) to (0.5,0.15) to (T1.sec2);
\draw [gray] (5.5,-0.15) to (D3) to (0.5,-0.15) to (T1.sec3);
\draw [blue] (5.5,-0.45) to (D4) to (0.5,-0.45) to (T1.sec4);

%neutre/terre

\node (RN) [resistor, label=$R_B$, rotate=90, tiny circuit symbols] at (0,-2.7) {};
\node (G1) [tlground] at (0,-3.9) {};
\draw [green!] (G1) to node {} (RN) ; 
\draw [green!] (RN) to (0,-0.5) to node {} (T1.sec4) ; 
\draw [dashed, yellow!] (G1) to node {} (RN) ;
\draw [dashed, yellow!] (RN) to (0,-0.5) to node {} (T1.sec4) ;

\node (G1) [tlground] at (0,-3.9) {};
\node (T1) [oosourcetransshape, prim=delta,sec=wye] at (0,0) {};

\node (RT) [resistor, label=$R_A$, rotate=90, tiny circuit symbols] at (2.5,-2.7) {};
\node (G2) [tlground] at (2.5,-3.9) {};
\draw [green!] (RT) to (G2); 
\draw [dashed, yellow!] (RT) to (G2);
\node (G2) [tlground] at (2.5,-3.9) {};

%appareil 1

\node (C1) [circ, scale=0.5] at (2.5,-1.8) {};
\node (C2) [circ, scale=0.5] at (2.4,0.45) {};
\node (C3) [circ, scale=0.5] at (2,-0.45) {};
\node (L) [bulb, info=1, rotate=270] at (2.4,-1.5) {};

\draw [green!] (C1) to node {} (RT); 
\draw [dashed, yellow!] (C1) to node {} (RT); 

\draw (2.1,-1.2) rectangle (2.7,-1.8);
\draw [brown] (C2) to node {} (L);
\draw [blue] (C3) to (2,-2) to (2.4,-2) to node {} (L);

%appareil 2

\node (C5) [circ, scale=0.5] at (3.7,-1.8) {};
\node (C6) [circ, scale=0.5] at (3.6,0.45) {};
\node (C7) [circ, scale=0.5] at (3.2,-0.45) {};
\node (C8) [circ, scale=0.5] at (2.5,-2.1) {};
\node (L) [bulb, info=2, rotate=270] at (3.6,-1.5) {};

\draw [green!] (C8) -| (C5); 
\draw [dashed, yellow!] (C8) -| (C5); 

\draw (3.3,-1.2) rectangle (3.9,-1.8);
\draw [brown] (C6) to node {} (L);
\draw [blue] (C7) to (3.2,-2) to (3.6,-2) to node {} (L);

%appareil 3

\node (C9) [circ, scale=0.5] at (4.9,-1.8) {};
\node (C10) [circ, scale=0.5] at (4.8,0.45) {};
\node (C11) [circ, scale=0.5] at (4.4,-0.45) {};
\node (C12) [circ, scale=0.5] at (2.5,-2.2) {};
\node (L) [bulb, info=3, rotate=270] at (4.8,-1.5) {};

\draw [green!] (C12) -| (C9); 
\draw [dashed, yellow!] (C12) -| (C9); 

\draw (4.5,-1.2) rectangle (5.1,-1.8);
\draw [brown] (C10) to node {} (L);
\draw [blue] (C11) to (4.4,-2) to (4.8,-2) to node {} (L);
%chemin courant

\callout{1,-2}{\cstep\label{pas:1}}{0.2,-3.8};
\callout{-0.5,-1}{\cstep\label{pas:2}}{-0.1,-2.4};
\callout{1.5,-3}{\cstep\label{pas:3}}{2.2,-3.8};
\callout{4,-2.6}{\cstep\label{pas:4}}{2.6,-2.6};
\startcstep %remet les compteurs des légendes en pastille à zéro
\end{circuitikz}
%\end{document}

