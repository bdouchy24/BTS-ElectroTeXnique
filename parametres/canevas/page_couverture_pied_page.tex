%--------------------------------------
%CANEVAS
%--------------------------------------

%utiliser les environnement \begin{comment} \end{comment} pour mettre en commentaire le préambule une fois la programmation appelée dans le document maître (!ne pas oublier de mettre en commentaire \end{document}!)

%\begin{comment}

\documentclass[a4paper, 11pt, twoside, fleqn]{memoir}

\usepackage{AOCDTF}

%--------------------------------------
%CANEVAS
%--------------------------------------

\newcommand\BoxColor{\ifcase\thechapshift blue!30\or brown!30\or pink!30\or cyan!30\or green!30\or teal!30\or purple!30\or red!30\or olive!30\or orange!30\or lime!30\or gray!\or magenta!30\else yellow!30\fi} %définition de la couleur des marqueurs de chapitre

\newcounter{chapshift} %compteur de chapitre du marqueur de chapitre
\addtocounter{chapshift}{-1}
	
\newif\ifFrame %instruction conditionnelle pour les couleurs des pages
\Frametrue

\pagestyle{plain}

% the main command; the mandatory argument sets the color of the vertical box
\newcommand\ChapFrame{%
\AddEverypageHook{%
\ifFrame
\ifthenelse{\isodd{\value{page}}}
  {\backgroundsetup{contents={%
  \begin{tikzpicture}[overlay,remember picture]
  \node[
  	rounded corners=3pt,
    fill=\BoxColor,
    inner sep=0pt,
    rectangle,
    text width=1.5cm,
    text height=5.5cm,
    align=center,
    anchor=north west
  ] 
  at ($ (current page.north west) + (-0cm,-2*\thechapshift cm) $) %nombre négatif = espacement des marqueurs entre les différents chapitres (à régler en fin de rédaction) (4.5cm vaut un espacement équivalement à la hauteur du marqueur, une page peut en contenir 6 avec cet espacement-la mais il est le plus équilibré)
    {\rotatebox{90}{\hspace*{.5cm}%
      \parbox[c][1.2cm][t]{5cm}{%
        \raggedright\textcolor{black}{\sffamily\textbf{\leftmark}}}}};
  \end{tikzpicture}}}
  }
  {\backgroundsetup{contents={%
  \begin{tikzpicture}[overlay,remember picture]
  \node[
  	rounded corners=3pt,
    fill=\BoxColor,
    inner sep=0pt,
    rectangle,
    text width=1.5cm,
    text height=5.5cm,
    align=center,
    anchor=north east
  ] 
  at ($ (current page.north east) + (-0cm,-2*\thechapshift cm) $) %nombre négatif = espacement des marqueurs entre les différents chapitres (à régler en fin de rédaction) (4.5cm vaut un espacement équivalement à la hauteur du marqueur, une page peut en contenir 6 avec cet espacement-la mais il est le plus équilibré)
    {\rotatebox{90}{\hspace*{.5cm}%
      \parbox[c][1.2cm][t]{5cm}{%
        \raggedright\textcolor{black}{\sffamily\textbf{\leftmark}}}}};
  \end{tikzpicture}}}%
  }
  \BgMaterial%
  \fi%
}%
  \stepcounter{chapshift}
}

\renewcommand\chaptermark[1]{\markboth{\thechapter.~#1}{}} %redéfinition du marqueur de chapitre pour ne contenir que le titre du chapitre %à personnaliser selon le nombre de chapitre dans le cours

%--------------------------------------
%corps du document
%--------------------------------------

\begin{document} %corps du document
	\openleft %début de chapitre à gauche

%\end{comment}

\pagestyle{empty}

\begingroup

%--------------------------------------
%arrière-plan
%--------------------------------------
	\begin{tikzpicture}[remember picture, overlay]
		\begin{scope}[shift={(current page.south west)}]
			\draw[draw=lightgray,fill=lightgray] (0,0) rectangle (21,30);
		\end{scope}
	\end{tikzpicture}

%--------------------------------------
%bandeau vertical
%--------------------------------------

	\begin{tikzpicture}[remember picture, overlay]
		\begin{scope}[shift={(current page.south west)}]
			\draw[draw=orangelogo,fill=orangelogo, opacity=0.7] (0,0) rectangle (4,30);
		\end{scope}
	\end{tikzpicture}

%--------------------------------------
%texte dans bandeau vertical
%--------------------------------------
  
	\begin{tikzpicture}[remember picture, overlay]
	\begin{scope}[shift={(current page.south west)}]
			\draw (1.8,1) node [right, rotate=90, opacity=0.7] {\sffamily\textbf{\Large\color{bleulogo}M\'ETIERS DES TECHNOLOGIES ASSOCI\'EES}};
			\draw (2.5,1) node [right, rotate=90, opacity=0.7] {\sffamily\textbf{\Large\color{bleulogo}ASSOCIATION OUVRIÈRE DES COMPAGNONS DU DEVOIR ET DU TOUR DE FRANCE}};
			\draw (3.4,29) node [left, rotate=90, opacity=0.7] {\sffamily\textls[200]{\Huge\textbf{\color{bleulogo}BTS \'ELECTROTECHNIQUE}}};
		\end{scope}
	\end{tikzpicture}

%--------------------------------------
%titre 
%--------------------------------------

	\begin{tikzpicture}[remember picture, overlay]
		\begin{scope}[shift={(current page.north west)}]
		%\draw (5,-11) node [right] {\sffamily\textls[150]{\Huge\textbf{\color{darkgray}\textsc{Matière ligne1}}}}; 	%ligne 1 long intitulé de la matière
		\draw (5,-12.5) node [right, opacity=0.7] {\sffamily\textls[150]{\Huge\textbf{\color{bleulogo}Matière}}}; %intitulé de la matière
		\draw (20,-14.5) node [left, opacity=0.7] {\sffamily\textls[150]{\Huge\textbf{\color{bleulogo}Cours}}}; %intitulé du cours
		%\draw (20,-16) node [left] {\sffamily\textls[150]{\Huge\textbf{\color{darkgray}\textsc{Cours ligne1}}}};  %ligne 1 long intitulé du cours
		\draw[very thick, orangelogo, opacity=0.7] (5,-13.5) -- (20,-13.5);
		\end{scope}
	\end{tikzpicture}

%--------------------------------------
%logo
%--------------------------------------

	\begin{tikzpicture}[remember picture, overlay]
		\begin{scope}[shift={(current page.north west)}]
			\draw (12.5,-5) %(12.5, -4) si long intitulé du chapitre
			node [opacity=0.3] {\includegraphics[scale=0.3]{logo_compagnons}};
			\draw (12.5,-21) %(12.5, -22) si long intitulé du chapitre
			node [opacity=0.3] {\includegraphics[scale=0.7]{logo_metier}};
			\draw (5,-28) node [right, opacity=0.7] {\includegraphics{pied_page_compagnons}};
		\end{scope}
	\end{tikzpicture}

\endgroup

\null\cleardoublepage

\end{document}