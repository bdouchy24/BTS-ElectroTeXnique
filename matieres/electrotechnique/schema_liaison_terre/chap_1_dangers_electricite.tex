%--------------------------------------
%ELECTROTECHNIQUE - SCHEMA DE LIAISON A LA TERRE
%--------------------------------------

%utiliser les environnement \begin{comment} \end{comment} pour mettre en commentaire le préambule une fois la programmation appelée dans le document maître (!ne pas oublier de mettre en commentaire \end{document}!)

%\begin{comment}

\documentclass[a4paper, 11pt, twoside, fleqn]{memoir}

\usepackage{AOCDTF}

%--------------------------------------
%CANEVAS
%--------------------------------------

\newcommand\BoxColor{\ifcase\thechapshift blue!30\or brown!30\or pink!30\or cyan!30\or green!30\or teal!30\or purple!30\or red!30\or olive!30\or orange!30\or lime!30\or gray!\or magenta!30\else yellow!30\fi} %définition de la couleur des marqueurs de chapitre

\newcounter{chapshift} %compteur de chapitre du marqueur de chapitre
\addtocounter{chapshift}{-1}
	
\newif\ifFrame %instruction conditionnelle pour les couleurs des pages
\Frametrue

\pagestyle{plain}

% the main command; the mandatory argument sets the color of the vertical box
\newcommand\ChapFrame{%
\AddEverypageHook{%
\ifFrame
\ifthenelse{\isodd{\value{page}}}
  {\backgroundsetup{contents={%
  \begin{tikzpicture}[overlay,remember picture]
  \node[
  	rounded corners=3pt,
    fill=\BoxColor,
    inner sep=0pt,
    rectangle,
    text width=1.5cm,
    text height=5.5cm,
    align=center,
    anchor=north west
  ] 
  at ($ (current page.north west) + (-0cm,-2*\thechapshift cm) $) %nombre négatif = espacement des marqueurs entre les différents chapitres (à régler en fin de rédaction) (4.5cm vaut un espacement équivalement à la hauteur du marqueur, une page peut en contenir 6 avec cet espacement-la mais il est le plus équilibré)
    {\rotatebox{90}{\hspace*{.5cm}%
      \parbox[c][1.2cm][t]{5cm}{%
        \raggedright\textcolor{black}{\sffamily\textbf{\leftmark}}}}};
  \end{tikzpicture}}}
  }
  {\backgroundsetup{contents={%
  \begin{tikzpicture}[overlay,remember picture]
  \node[
  	rounded corners=3pt,
    fill=\BoxColor,
    inner sep=0pt,
    rectangle,
    text width=1.5cm,
    text height=5.5cm,
    align=center,
    anchor=north east
  ] 
  at ($ (current page.north east) + (-0cm,-2*\thechapshift cm) $) %nombre négatif = espacement des marqueurs entre les différents chapitres (à régler en fin de rédaction) (4.5cm vaut un espacement équivalement à la hauteur du marqueur, une page peut en contenir 6 avec cet espacement-la mais il est le plus équilibré)
    {\rotatebox{90}{\hspace*{.5cm}%
      \parbox[c][1.2cm][t]{5cm}{%
        \raggedright\textcolor{black}{\sffamily\textbf{\leftmark}}}}};
  \end{tikzpicture}}}%
  }
  \BgMaterial%
  \fi%
}%
  \stepcounter{chapshift}
}

\renewcommand\chaptermark[1]{\markboth{\thechapter.~#1}{}} %redéfinition du marqueur de chapitre pour ne contenir que le titre du chapitre %à personnaliser selon le nombre de chapitre dans le cours

%--------------------------------------
%corps du document
%--------------------------------------

\begin{document} %corps du document
	\openleft %début de chapitre à gauche

%\end{comment}

\chapter{Les dangers de l'électricité}
\ChapFrame %appel du marqueur de chapitre



\section{Catégories de tension}


\begin{table}[h]
\caption{Domaines de tensions\label{tab:categories_tension}}
\begin{threeparttable} %note dans tableau
\begin{tabularx}{\textwidth}{X l J J}
\toprule
\multicolumn{2}{c}{\thead{Domaine de tension}}	& \multicolumn{1}{c}{\thead[c]{Courant alternatif\tnote{\ 1}}}		&  \multicolumn{1}{c}{\thead[c]{Courant continu}} \\
\midrule
Très Basse Tension										& TBT		& U \leq 50\volt								& U \leq 120\volt \\
Basse Tension												& BT			& 50\volt < U \leq 1000\volt			& 120\volt < U \leq 1500\volt \\
\multirow[t]{2}{*}{Haute Tension\tnote{2}}	& HTA		& 1000\volt < U \leq 50\kilo\volt		& 1500\volt < U \leq 75\kilo\volt \\
																	& HTB		& U > 50\kilo\volt							& U > 75\kilo\volt \\
\bottomrule
\end{tabularx}
\begin{tablenotes}
    \item[1] Tension nominale exprimée en \emph{valeur efficace} $U_n$\,;
    \item[2] Les basses tensions ne sont plus divisées en deux catégories depuis 2010, seule la haute tension conserve cette caractéristique.
\end{tablenotes}
\end{threeparttable}
\end{table}
	
\section{Action du courant électrique sur le corps humain}

La présence d'une tension électrique entraine toujours un risque de choc électrique mais il est peu aisé de déterminer un seuil de tension pour lequel le choc est dangereux car ce sont le \emph{courant} $I$ traversant le corps et la \emph{durée} $t$ du choc électrique qui permettent de déterminer la probabilité de décès
\begin{equa} %environnement pour créer des équations référencées avec légendes alignée sur &=
	\begin{align} 
		I &= \frac{116}{\sqrt{t}}
	\end{align}
\caption{Valeur statistique du courant entrainant la mort en fonction de la durée}
\label{eq:valeur_courant_mort}
\end{equa}

\begin{textvariables}
I						& courant électrique							& milliampère			& \milli\ampere					& Courant traversant le corps 	\\
t						& durée												& seconde					& \second							&	durée du choc électrique d'une durée ($8\milli\second < t \leq 5\second \nonumber$) \\
116					& constante										& / 							& 	/									&	constante empirique déterminée statistiquement \\
\end{textvariables}

\subsection{Effet du courant alternatif}

Les effets du courant alternatif entre \SI{15}{\hertz} et \SI{100}{\hertz} sont décrit en \autoref{fig:effets_courant_electrique_alternatif}. 

%--------------------------------------
%ELECTROTECHNIQUE - SCHEMA DE LIAISON A LA TERRE
%--------------------------------------

%utiliser les environnement \begin{comment} \end{comment} pour mettre en commentaire le préambule une fois la programmation appelée dans le document maître (!ne pas oublier de mettre en commentaire \end{document}!)

\begin{comment}

\documentclass[a4paper, 11pt, twoside, fleqn]{memoir}

\usepackage{AOCDTF}

%--------------------------------------
%CANEVAS
%--------------------------------------

\newcommand\BoxColor{\ifcase\thechapshift blue!30\or brown!30\or pink!30\or cyan!30\or green!30\or teal!30\or purple!30\or red!30\or olive!30\or orange!30\or lime!30\or gray!\or magenta!30\else yellow!30\fi} %définition de la couleur des marqueurs de chapitre

\newcounter{chapshift} %compteur de chapitre du marqueur de chapitre
\addtocounter{chapshift}{-1}
	
\newif\ifFrame %instruction conditionnelle pour les couleurs des pages
\Frametrue

\pagestyle{plain}

% the main command; the mandatory argument sets the color of the vertical box
\newcommand\ChapFrame{%
\AddEverypageHook{%
\ifFrame
\ifthenelse{\isodd{\value{page}}}
  {\backgroundsetup{contents={%
  \begin{tikzpicture}[overlay,remember picture]
  \node[
  	rounded corners=3pt,
    fill=\BoxColor,
    inner sep=0pt,
    rectangle,
    text width=1.5cm,
    text height=5.5cm,
    align=center,
    anchor=north west
  ] 
  at ($ (current page.north west) + (-0cm,-2*\thechapshift cm) $) %nombre négatif = espacement des marqueurs entre les différents chapitres (à régler en fin de rédaction) (4.5cm vaut un espacement équivalement à la hauteur du marqueur, une page peut en contenir 6 avec cet espacement-la mais il est le plus équilibré)
    {\rotatebox{90}{\hspace*{.5cm}%
      \parbox[c][1.2cm][t]{5cm}{%
        \raggedright\textcolor{black}{\sffamily\textbf{\leftmark}}}}};
  \end{tikzpicture}}}
  }
  {\backgroundsetup{contents={%
  \begin{tikzpicture}[overlay,remember picture]
  \node[
  	rounded corners=3pt,
    fill=\BoxColor,
    inner sep=0pt,
    rectangle,
    text width=1.5cm,
    text height=5.5cm,
    align=center,
    anchor=north east
  ] 
  at ($ (current page.north east) + (-0cm,-2*\thechapshift cm) $) %nombre négatif = espacement des marqueurs entre les différents chapitres (à régler en fin de rédaction) (4.5cm vaut un espacement équivalement à la hauteur du marqueur, une page peut en contenir 6 avec cet espacement-la mais il est le plus équilibré)
    {\rotatebox{90}{\hspace*{.5cm}%
      \parbox[c][1.2cm][t]{5cm}{%
        \raggedright\textcolor{black}{\sffamily\textbf{\leftmark}}}}};
  \end{tikzpicture}}}%
  }
  \BgMaterial%
  \fi%
}%
  \stepcounter{chapshift}
}

\renewcommand\chaptermark[1]{\markboth{\thechapter.~#1}{}} %redéfinition du marqueur de chapitre pour ne contenir que le titre du chapitre %à personnaliser selon le nombre de chapitre dans le cours

%--------------------------------------
%corps du document
%--------------------------------------

\begin{document} %corps du document
	\openleft %début de chapitre à gauche

\end{comment}

\begin{figure}[H]
\centering
\caption{Effets du courant alternatif sur le corps humain \label{fig:effets_courant_electrique_alternatif}}
\begin{tikzpicture}
%\DrawGrid{(-7,-5)}{(7,5)} %grille d'aide pour le placement des objets

\draw (-7,4) node [right, minimum width=2cm, text width=1.8cm] {\footnotesize{Intensité de contact $I_c$}};
\draw [->] (-5,2.2) -- (-5,4);
\draw (-5, 2) node {\vdots};
\draw (-5,-3.9) -- (-5,1.6);

\draw (-5.4,-3) node [left] {\SI{0,5}{\milli\ampere}};
\draw [->] (-5.2,-3) -- (-4.5,-3);
\draw (-4.2,-3) node [anchor=mid west, text width=7cm] (A) {Sensation de picotements, de très faible à faible};
\draw (7,-3.5) node [anchor=east] {\includegraphics[width=2cm]{sensation}};

\draw (-5.4,-2.6) node [left] {\SI{10}{\milli\ampere}};
\draw [->] (-5.2,-2.6) -- (-4.5,-2.6);
\draw (-4.2,-2.6) node [anchor=mid west, text width=7cm] {Seuil de non lâcher, paralysie musculaire};
\draw (3.2,-2.6) node [anchor=west] {\includegraphics[width=1.5cm]{tetanisation}};

\draw (-5.4,-1.8) node [left] {\SI{30}{\milli\ampere}};
\draw [->] (-5.2,-1.8) -- (-4.5,-1.8);
\draw (-4.2,-1.8) node [anchor=mid west, text width=7cm] {Seuil de paralysie respiratoire};
\draw (7,-1.8) node [anchor=east] {\includegraphics[width=1.5cm]{paralysie_poumon}};

\draw (-5.4,0.7) node [left] {\SI{75}{\milli\ampere}};
\draw [->] (-5.2,0.7) -- (-4.5,0.7);
\draw (-4.2,0.7) node [anchor=mid west, text width=7cm] {Seuil de fibriliation ventriculaire};
\draw (5.1,0.7) node {\includegraphics[width=1.5cm]{fibrillation}};

\draw (-5.4,3.1) node [left] {\SI{1}{\ampere}};
\draw [->] (-5.2,3.1) -- (-4.5,3.1);
\draw (-4.2,3.1) node [anchor=mid west, text width=7cm] {Arrêt du coeur};
\draw (5.2,3.1) node {\includegraphics[width=2.4cm]{arret_cardiaque}};

\end{tikzpicture}
\end{figure}

%\end{document}

\subsubsection{Cas particuliers}

Pour le courant alternatifs d'une fréquence supérieures à \SI{100}{\hertz} :

\begin{itemize}
\item Plus la fréquence du courant augmente, plus les risques de fibrillation ventriculaire diminue\,;
\item Plus la fréquence du courant augmente, plus les risques de brûlures augmentent\,;
\item Plus la fréquence du courant augmente, plus l'impédance du corps humain diminue\,;
\item Il est généralement considéré que les conditions de protection contre les contacts indirects sont identiques que ça soit sous une fréquence de \SI{50}{\hertz} (réseau électrique domestique en Europe) où \SI{400}{\hertz} (réseau électrique des bateaux, avions\ldots).
\end{itemize}

\subsection{Effet du courant continu}

Les effets du courant continus sont décrits en \autoref{fig:effets_courant_electrique_continu}.

%--------------------------------------
%ELECTROTECHNIQUE - SCHEMA DE LIAISON A LA TERRE
%--------------------------------------

%utiliser les environnement \begin{comment} \end{comment} pour mettre en commentaire le préambule une fois la programmation appelée dans le document maître (!ne pas oublier de mettre en commentaire \end{document}!)

\begin{comment}

\documentclass[a4paper, 11pt, twoside, fleqn]{memoir}

\usepackage{AOCDTF}

%--------------------------------------
%CANEVAS
%--------------------------------------

\newcommand\BoxColor{\ifcase\thechapshift blue!30\or brown!30\or pink!30\or cyan!30\or green!30\or teal!30\or purple!30\or red!30\or olive!30\or orange!30\or lime!30\or gray!\or magenta!30\else yellow!30\fi} %définition de la couleur des marqueurs de chapitre

\newcounter{chapshift} %compteur de chapitre du marqueur de chapitre
\addtocounter{chapshift}{-1}
	
\newif\ifFrame %instruction conditionnelle pour les couleurs des pages
\Frametrue

\pagestyle{plain}

% the main command; the mandatory argument sets the color of the vertical box
\newcommand\ChapFrame{%
\AddEverypageHook{%
\ifFrame
\ifthenelse{\isodd{\value{page}}}
  {\backgroundsetup{contents={%
  \begin{tikzpicture}[overlay,remember picture]
  \node[
  	rounded corners=3pt,
    fill=\BoxColor,
    inner sep=0pt,
    rectangle,
    text width=1.5cm,
    text height=5.5cm,
    align=center,
    anchor=north west
  ] 
  at ($ (current page.north west) + (-0cm,-2*\thechapshift cm) $) %nombre négatif = espacement des marqueurs entre les différents chapitres (à régler en fin de rédaction) (4.5cm vaut un espacement équivalement à la hauteur du marqueur, une page peut en contenir 6 avec cet espacement-la mais il est le plus équilibré)
    {\rotatebox{90}{\hspace*{.5cm}%
      \parbox[c][1.2cm][t]{5cm}{%
        \raggedright\textcolor{black}{\sffamily\textbf{\leftmark}}}}};
  \end{tikzpicture}}}
  }
  {\backgroundsetup{contents={%
  \begin{tikzpicture}[overlay,remember picture]
  \node[
  	rounded corners=3pt,
    fill=\BoxColor,
    inner sep=0pt,
    rectangle,
    text width=1.5cm,
    text height=5.5cm,
    align=center,
    anchor=north east
  ] 
  at ($ (current page.north east) + (-0cm,-2*\thechapshift cm) $) %nombre négatif = espacement des marqueurs entre les différents chapitres (à régler en fin de rédaction) (4.5cm vaut un espacement équivalement à la hauteur du marqueur, une page peut en contenir 6 avec cet espacement-la mais il est le plus équilibré)
    {\rotatebox{90}{\hspace*{.5cm}%
      \parbox[c][1.2cm][t]{5cm}{%
        \raggedright\textcolor{black}{\sffamily\textbf{\leftmark}}}}};
  \end{tikzpicture}}}%
  }
  \BgMaterial%
  \fi%
}%
  \stepcounter{chapshift}
}

\renewcommand\chaptermark[1]{\markboth{\thechapter.~#1}{}} %redéfinition du marqueur de chapitre pour ne contenir que le titre du chapitre %à personnaliser selon le nombre de chapitre dans le cours

%--------------------------------------
%corps du document
%--------------------------------------

\begin{document} %corps du document
	\openleft %début de chapitre à gauche

\end{comment}

\begin{figure}[H]
\centering
\caption{Effets du courant continu sur le corps humain \label{fig:effets_courant_electrique_continu}}
\begin{tikzpicture}
%\DrawGrid{(-7,-2)}{(7,5)} %grille d'aide pour le placement des objets

\draw (-7,4) node [right, minimum width=2cm, text width=1.8cm] {\footnotesize{Intensité de contact $I_c$}};
\draw [->] (-5,-1) -- (-5,4);

\draw (-5.4,0) node [left] {\SI{2}{\milli\ampere}};
\draw [->] (-5.2,0) -- (-4.5,0);
\draw (-4.2,0) node [anchor=mid west, text width=7cm] (A) {Sensation de picotements, de très faible à faible};
\draw (6,0) node [anchor=east] {\includegraphics[width=2cm]{sensation}};

\draw (-5.4,1.5) node [left, text width=1cm, align=right] {Non défini};
\draw [->] (-5.2,1.5) -- (-4.5,1.5);
\draw (-4.2,1.5) node [anchor=mid west, text width=7cm] {Seuil de non lâcher, paralysie musculaire};
\draw (3.2,1.5) node [anchor=west] {\includegraphics[width=1.5cm]{tetanisation}};

\draw (-5.4,3) node [left] {\SI{130}{\milli\ampere}};
\draw [->] (-5.2,3) -- (-4.5,3);
\draw (-4.2,3) node [anchor=mid west, text width=7cm] {Seuil de fibriliation ventriculaire};
\draw (5.1,3) node {\includegraphics[width=1.5cm]{fibrillation}};

\end{tikzpicture}
\end{figure}

%\end{document}

\begin{itemize}
\item Il est moins difficile de lâcher les parties tenues à la main sous un courant continu\,;
\item Le seuil de fibrillation ventriculaire est plus élevé.
\end{itemize}

\section{Paramètres influençant les risques électriques}

L'intensité de contact $I_c$, la durée de contact $t$, la tension de contact $U_c$ et la résistance du corps humain $R$ sont autant de paramètres à prendre en compte lors de l'évaluation des risques électriques.

\begin{tikzpicture}
\pgfplotsset{grid style={dashed}} %style de la grille
\begin{axis}[
grid=major, %grille
xmode=log, xmin=0.1, xmax=5000,
ymode=log, ymin=10, ymax=10000,

]


\end{document}